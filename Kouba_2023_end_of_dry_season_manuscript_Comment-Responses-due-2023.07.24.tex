%% Copernicus Publications Manuscript Preparation Template for LaTeX Submissions
%% ---------------------------------
%% This template should be used for copernicus.cls
%% The class file and some style files are bundled in the Copernicus Latex Package, which can be downloaded from the different journal webpages.
%% For further assistance please contact Copernicus Publications at: production@copernicus.org
%% https://publications.copernicus.org/for_authors/manuscript_preparation.html

%% copernicus_rticles_template (flag for rticles template detection - do not remove!)

%% Please use the following documentclass and journal abbreviations for discussion papers and final revised papers.

%% 2-column papers and discussion papers
\documentclass[hess, manuscript]{copernicus}



%% Journal abbreviations (please use the same for preprints and final revised papers)

% Advances in Geosciences (adgeo)
% Advances in Radio Science (ars)
% Advances in Science and Research (asr)
% Advances in Statistical Climatology, Meteorology and Oceanography (ascmo)
% Annales Geophysicae (angeo)
% Archives Animal Breeding (aab)
% Atmospheric Chemistry and Physics (acp)
% Atmospheric Measurement Techniques (amt)
% Biogeosciences (bg)
% Climate of the Past (cp)
% DEUQUA Special Publications (deuquasp)
% Drinking Water Engineering and Science (dwes)
% Earth Surface Dynamics (esurf)
% Earth System Dynamics (esd)
% Earth System Science Data (essd)
% E&G Quaternary Science Journal (egqsj)
% EGUsphere (egusphere) | This is only for EGUsphere preprints submitted without relation to an EGU journal.
% European Journal of Mineralogy (ejm)
% Fossil Record (fr)
% Geochronology (gchron)
% Geographica Helvetica (gh)
% Geoscience Communication (gc)
% Geoscientific Instrumentation, Methods and Data Systems (gi)
% Geoscientific Model Development (gmd)
% History of Geo- and Space Sciences (hgss)
% Hydrology and Earth System Sciences (hess)
% Journal of Bone and Joint Infection (jbji)
% Journal of Micropalaeontology (jm)
% Journal of Sensors and Sensor Systems (jsss)
% Magnetic Resonance (mr)
% Mechanical Sciences (ms)
% Natural Hazards and Earth System Sciences (nhess)
% Nonlinear Processes in Geophysics (npg)
% Ocean Science (os)
% Polarforschung - Journal of the German Society for Polar Research (polf)
% Primate Biology (pb)
% Proceedings of the International Association of Hydrological Sciences (piahs)
% Safety of Nuclear Waste Disposal (sand)
% Scientific Drilling (sd)
% SOIL (soil)
% Solid Earth (se)
% The Cryosphere (tc)
% Weather and Climate Dynamics (wcd)
% Web Ecology (we)
% Wind Energy Science (wes)

% Pandoc citation processing

% The "Technical instructions for LaTex" by Copernicus require _not_ to insert any additional packages.
% 
% tightlist command for lists without linebreak
\providecommand{\tightlist}{%
  \setlength{\itemsep}{0pt}\setlength{\parskip}{0pt}}


%
\begin{document}


\title{Seasonal prediction of end-of-dry season watershed behavior in a
highly interconnected alluvial watershed, northern California}


\Author[1]{Claire}{Kouba}
\Author[1]{Thomas}{Harter}


\affil[1]{Department of Land, Air and Water Resources, University of
California, Davis, One Shields Avenue, Davis, CA, United States}

\runningtitle{Seasonal prediction of end-of-dry season watershed
behavior in a highly interconnected alluvial watershed, northern
California}

\runningauthor{Kouba and Harter}


\correspondence{Claire\ Kouba\ (cmkouba@ucdavis.edu)}



\received{}
\pubdiscuss{} %% only important for two-stage journals
\revised{}
\accepted{}
\published{}

%% These dates will be inserted by Copernicus Publications during the typesetting process.


\firstpage{1}

\maketitle


\begin{abstract}
In undammed watersheds in Mediterranean climates, the timing and
abruptness of the transition from the dry season to the wet season have
major implications for aquatic ecosystems. Of particular concern in many
coastal areas is whether this transition can provide sufficient flows at
the right time to allow passage for spawning anadromous fish, which is
determined by dry season baseflow rates and the timing of the onset of
the rainy season. In (semi-) ephemeral watershed systems, these
functional flows also dictate the timing of full reconnection of the
stream system. In this study, we propose methods to predict,
approximately five months in advance, two key hydrologic metrics in the
undammed rural Scott River watershed (HUC8 18010208) in northern
California. Both metrics are intended to quantify the transition from
the dry to the wet season, to characterize the severity of a dry year
and support seasonal adaptive management. The first metric is the
minimum 30-day dry season baseflow volume, \(V_{min}\), which occurs at
the end of the dry season (September-October) in this Mediterranean
climate. The second metric is the cumulative precipitation, starting
Sept.~1st, necessary to bring the watershed to a ``full'' or
``spilling'' condition (i.e.~initiate the onset of wet season storm- or
baseflows) after the end of the dry season, referred to here as
\(P_{spill}\). As potential predictors of these two values, we assess
maximum snowpack, cumulative precipitation, the timing of the snowpack
and precipitation, spring groundwater levels, spring river flows,
reference ET, and a subset of these metrics from the previous water
year. We find that, though many of these predictors are correlated with
the two metrics of interest, of the predictors considered here, the best
prediction for both metrics is a linear combination of the maximum
snowpack water content and total October-April precipitation. These two
linear models could reproduce historic values of \(V_{min}\) and
\(P_{spill}\) with an average model error (RMSE) of 1.4
Mm\textsuperscript{3} / 30 days (19.4 cfs) and 25.4 mm (1 inch),
corresponding to 49\% and 37\% of mean observed values, respectively.
Although these predictive indices could be used by governance entities
to support local water management, careful consideration of baseline
conditions used as a basis for prediction is necessary.
\end{abstract}


\copyrightstatement{The authors retain copyright for this publication.}


\section{Introduction}

In regions that experience periodic drought, such as the western United
States, spatially distributed indices summarizing hydroclimate or
surface water supply conditions are often critical decision-support
tools for water managers \citep[e.g.,][]{Garen1993}. An index can be
forward-looking, such as those that forecast near-term seasonal water
supplies \citep[e.g.,][]{Null2013, Verley2020}, or backward-looking,
such as ones that evaluate drought severity
\citep[e.g.,][]{Palmer1965, Guttman1998a, McKee1993, Wilhite1985, Wilhite2000}.
In many western states, forward-looking seasonal indices are used
extensively by water agencies to inform local adaptive management
decisions, e.g.~in Colorado \citeyearpar[CDWR][]{ColoradoDWR2023}, Idaho
\citeyearpar[NRCS][]{NRCS2023} and California \citep{Null2013}. In
California the principal examples are the Sacramento Valley Index (SVI)
and San Joaquin Index (SJI), which are seasonal forecasts used to
determine water allocations from these watersheds through the State
Water Project \citeyearpar[DWR][]{DWR2022}. The state has more recently
published a retroactive categorical water year type (WYT) dataset for
sub-county level regions throughout California
\citeyearpar[DWR][]{DWR2021a}.

These summary indices provide broad characterizations of anticipated or
historically available water supplies. More detailed characteristics,
including the hydrologic effects of water year type, climate change,
human water use, and other factors, can be identified using the
functional flows approach
\citep[e.g.,][]{Poff1997, Bunn2002, Poff2010, Wheeler2018}. The flows
are ``functional'' because they serve an ecological purpose, such as wet
season flood flows, needed to disperse cottonwood seeds
\citep{Mahoney1998} and fall pulse flows, needed to provide passage for
spawning fall-run anadromous fish \citep{Moyle2002a}. A
California-specific functional flows framework has been developed to
assess the degree of hydrologic alteration between current and
unimpaired conditions \citep{Yarnell2020, Patterson2020}.

Unlike forward-looking indices such as the SVI and SJI, the functional
flows approach \citep[e.g.][]{Yarnell2020} does not provide numerical
flow predictions from recent seasonal hydrologic datasets. The existing
forward-looking indices developed for Mediterranean climates (such as
the ones used in California, Idaho and Colorado, referenced above) are
particularly useful in stream systems with significant managed surface
water storage, capable of bridging the temporal gap between the timing
of precipitation and water use \citeyearpar[e.g., DWR][]{DWR2023}. Under
Mediterranean climate conditions, the completion of the wet season
defines the total available water supply, which is then managed through
reservoirs for supply deliveries throughout the irrigation season
\citeyearpar[e.g., CDWR][]{ColoradoDWR2023}. However, such indices have
not been employed in basins without managed surface water storage.
Neither have such indices been developed specifically to manage
environmental (instream) flow protection. Finally, such forward-looking
indices to inform adaptive water supply management are lacking in
smaller-scale geographic settings, especially for the mixed rain and
snow-fed stream category of Patterson et al., 2020.

Here, we outline and test a novel approach to developing a
forward-looking index that may be useful to inform water management
decisions targeting environmental instream flow management in mixed rain
and snow-fed watersheds without managed surface water storage under
Mediterranean climates. Environmentally, a critical period in such
systems is the summer baseflow period at the end of the dry season,
bracketed by the onset of winter stormflow \citep{Peek2022}. Summer
baseflow conditions are primarily an expression of underlying
groundwater storage, fed in turn by recharge from winter and spring
precipitation, snowmelt, and runoff \citep{Tarboton2003}. During this
period, native fish, particularly anadromous and/or salmonid fish, are
highly vulnerable to below average low flow conditions
\citep[e.g.,][]{VanKirk2008a}. Due to lack of surface water storage, to
manage environmental flows, such basins may seek early protective water
management decisions. These decisions must be made prior to the onset of
the irrigation season, but following the (near) completion of the wet
season to quantify the key annual water input to a Mediterranean
watershed.

We utilize the Scott River watershed in northern California as an
example Mediterranean climate stream system (without surface water
storage) to outline our approach and to evaluate whether a statistically
significant, forward looking index can be defined to support
environmental water management. Three periods of water use and climate
forces have been proposed for the Scott River \citep[e.g.,
by][]{Pyschik2022}: Eras 1, 2, and 3, ranging from 1942-1976, 1977-2000,
and 2001-2021, respectively. These eras reflect changes in human
management, such as the widespread installation of groundwater pumps in
the region in the late 1970s \citep{Tolley2019}; and climate conditions,
such as the shift in the Pacific Decadal Oscillation in the late 1970s
\citep{Francis1998} or the onset of a two-decade abnormally dry period
in 2000 \citep{Williams2020}. These overlapping changes make it
difficult to identify the cause of decade-scale changes in regional
hydrology; therefore, the proposed predictive method of hydrologic
behavior is agnostic as to the mechanism linking the predictors and
hydrologic response.

We first review the hydrologic indices and methods currently used in
decision-making, such as agricultural cropping choices or regulatory
water use restrictions, and propose two additional decision-support
metrics, both designed as quantitative forecasts. We additionally
explore the significance of each metric in the context of functional
flows. The first metric is \(V_{min}\), the minimum 30-day dry season
baseflow volume in a given water year, which typically occurs in
September or October. The second is a prediction of the cumulative
rainfall needed to wet up the watershed after the dry season such that
subsequent rainfall results in clear runoff, or storm surge, events.
This cumulative precipitation depth is referred to as \(P_{spill}\).
Both of these metrics have significance for environmental flows and
could support near-term (seasonal) adaptive management, similar to the
SVI and SJI in California's Central Valley. Specifically, the magnitude
of the minimum baseflow rate sets the spatial extent of the aquatic
ecosystem during the dry season and influences rearing conditions for
oversummering juvenile salmonids \citep{Gorman2016}. \(P_{spill}\),
meanwhile, is intuitively related to the timing of flows necessary for
fall-run salmon passage: under equivalent fall rainfall, a greater
amount of rain needed to generate stormflow would be associated with a
prolonged dry season. This type of prolonged dry season has delayed
salmon access to spawning habitat in recent years
\citeyearpar[CDFW][]{CDFW2015a}. After defining and developing seasonal
predictions for \(V_{min}\) and \(P_{spill}\), we then evaluate trends
over time and consider the effects that climate change and changing
water use patterns may have on the metrics considered in this study, and
the decisions they support.

\section{Methods}

In this study we used linear regression modeling to predict watershed
behavior at the end of the dry season (the response) using data
available the previous spring (the predictors). The Scott River
watershed (\autoref{fig:watershed_fig_ch3}) has a snow-influenced
Mediterranean climate, giving the river's annual hydrograph a
characteristic high-flow season during the rainy winters, a gradual flow
recession in the spring-summer as the snowpack melts, and a low-flow dry
season after the snowpack is depleted (e.g.,
\autoref{fig:watershed_4states_illustration}). In the U.S. Geological
Survey (USGS) National Hydrography Dataset, the Scott River watershed is
denoted with the 8-digit Hydrologic Unit Code (HUC8) 18010208. Annual
demand for agricultural and domestic use (estimated at 23 and 1.3
thousand acre-feet, respectively) \citeyearpar[DWR][]{DWR2004} are
relatively stable in the Scott River system (although some reports of
dry wells occur in dry years) \citeyearpar[Siskiyou
County][]{SiskiyouCounty2021}. A key management challenge is persistent
low environmental flows during the dry season baseflow period. In dry
years, the lowest annual flowrates can overlap with the spawning periods
for fall-run anadromous fish, potentially restricting fish passage and
imperiling the long-term viability of the Scott River fishery
\citeyearpar[Siskiyou County][]{SiskiyouCounty2021}. Post-1970s minimum
dry season baseflows have been lower than pre-1977, and very low minima
(\textless{} 10 cfs or 0.7 Mm\textsuperscript{3} / 30 days) have been
more frequent in the past two decades (see Results, Section 3.2.1),
making the management of these flows more urgent.

This study focuses on the transition from the dry season to the wet
season, which at times can straddle the conventional water year boundary
of October 1st, and cumulative precipitation is used both as a predictor
and as a response variable (\(P_{spill}\)). When it is a predictor, a
traditional October 1st start date is used and it is summed as the
cumulative precipitation of October-April, to facilitate an end-of-April
prediction of fall conditions. When it is the response variable, to
capture uncommon September precipitation, cumulative precipitation is
counted starting on September 1st of the preceding water year. This
September 1st start date is also used in some graphs of climate and flow
data (e.g.~in Section 3.2 below), to establish and visualize baseline
dry season conditions. Additionally, all flows in this study are
observed or simulated at the USGS Fort Jones streamflow gauge (station
11519500), a key monitoring location downstream of nearly all water use
and cultivated land in the HUC8 watershed
(\autoref{fig:watershed_fig_ch3}), with an observation record covering
water years 1942-2021.

\begin{figure}
\includegraphics[width=1\linewidth]{f01} \caption{\label{fig:watershed_fig_ch3} Scott River HUC8 watershed and groundwater basin boundaries, stream network, and key monitoring locations: the Fort Jones stream gauge (USGS ID 11519500), weather stations, snow observation locations, and a CIMIS station (used to estimate reference evapotranspiration). Selected locations are highlighted with an enlarged symbol and an abbreviated label.}\label{fig:watershed_fig_ch3}
\end{figure}

\begin{figure}
\includegraphics[width=1\linewidth]{f02} \caption{\label{fig:watershed_4states_illustration} Illustration of four categories of Scott River watershed behavior. The hydrograph in the highlighted periods demonstrates the following watershed behavior: A, dry season baseflow -- watershed draining from a medium-to-low storage level; B, moderate flow increase  -- muted flow surge response to new precipitation; C, winter baseflow and early spring recession -- watershed draining from a high storage level; and D, winter stormflow -- rapid flow surge response to new precipitation (storm spikes).}\label{fig:watershed_4states_illustration}
\end{figure}

\subsection{Scott River watershed precipitation-runoff behavior}

To establish the context and hydrologic relevance of the two proposed
predictive indices \(V_{min}\) and \(P_{spill}\), a brief description of
the behavior of the watershed is necessary. In an undammed catchment,
the runoff response to one (or a series of) precipitation event(s) is
dependent on multiple factors, including antecedent soil moisture
conditions, the intensity and magnitude of the precipitation, and on the
dampening and delay of runoff; the latter is due to interflow, snow
storage and recharge to groundwater storage that later returns as stream
baseflow \citep{Tarboton2003}. A threshold runoff response to individual
storm events has been observed at the hillslope scale where soil
directly overlays (relatively) impermeable bedrock: absent significant
aquifer storage, subsurface flow increases dramatically after a
quantifiable threshold of precipitation is reached
\citep{Tromp-VanMeerveld2006}. The proposed mechanism is the filling and
connecting of various distributed storage volumes, such as soil pores
and microtopographic relief in the bedrock surface
\citep{Tromp-VanMeerveld2006}. Recently this concept has been extended
to the watershed or basin scale: relative to the beginning of a storm
event, a much higher flow response is possible only when a critical
number of storage volumes throughout a basin fill to a threshold level
and become connected \citep{McDonnell2021}.

In this study we expand this concept of a basin-scale, threshold-based
runoff response to the temporal scale of a season, rather than a single
storm event. In this framework the condition of the Scott River
watershed, as measured at the basin scale using the Fort Jones stream
gauge, can be classified in four main categories. These categories are
distinguished by current precipitation conditions and the volumetric
proportion of the hydrologically connected reservoirs that are full of
water (\autoref{tab:watershed_modes_tab}).

Water in the Scott River watershed is stored in five primary reservoirs
\citep{Harter2008}:

\begin{itemize}
\tightlist
\item
  snowpack
\item
  fractures in impermeable bedrock
\item
  soil moisture/subflow
\item
  the alluvial groundwater aquifer
\item
  streams and surface water bodies
\end{itemize}

Accumulating snowpack is present only in the mountainous areas of the
upper watershed, and limited to the winter and spring season. The
alluvial aquifer is present only within the bounds of the groundwater
basin underlying the flat valley floor; water stored in fractured rock
emerges as springs in the upper watershed \citep{Mack1958}
(\autoref{fig:watershed_fig_ch3}). In conditions with sufficiently high
soil water content or groundwater elevations, soil moisture/subflow and
groundwater become hydrologically connected to the surface water system.
Conversely, water in the snowpack and fractured rock reservoirs is not
hydrologically connected to major surface water bodies until it melts or
descends lower into the watershed, effectively passing through the
aquifer or soil to reach the stream. For convenience the soil
moisture/subflow and aquifer reservoirs will be referred to as
``connected'' storage. Storage in streams and surface water bodies is
negligible at temporal scales exceeding a few days.

\begin{table*}[t]
\caption{Schematic of watershed behavior and functional flow components occurring during the transition from the dry season to the wet season in a Mediterranean climate; the categories are illustrated in an example annual hydrograph in Figure 2. Water storage level refers to the relative water content of the soil and aquifer within the watershed.}
\label{tab:watershed_modes_tab}
\begin{tabular}{p{1.8cm} p{1.6cm} p{6.0cm} p{3.8cm}}
\tophline
Water storage level & New precip. occurring? & Flow behavior description & Relevant functional flows \\
\middlehline
 Low & No & (A) Watershed draining from a medium-to-low storage level (depleting snow storage filling groundwater storage via recharge; depleting groundwater storage) & Late spring recession and dry season baseflow \\ 
 \middlehline
Low & Yes & (B) Watershed filling from a low storage level, with muted response to new precipitation (hillslope interflow, landscape recharge, rapid streambed recharge into alluvial groundwater basin, especially near the mountain front - downstream from the bedrock-alluvium transition) & Fall pulse flow or small/slow post-dry-season flow increase \\
\middlehline
High & No & (C) Watershed draining from a high storage level (hillslope interflow, snowmelt, groundwater discharge to streams) & Winter baseflow and early spring recession \\
\middlehline
High & Yes & (D) Watershed spilling from a high storage level, with rapid response to new precipitation & Winter stormflow\\
\bottomhline
\end{tabular}
\belowtable{}
\end{table*}

\subsubsection{\texorpdfstring{Rainfall-runoff response, functional
flows and
\(Q_{spill}\)}{Rainfall-runoff response, functional flows and Q\_\{spill\}}}

In the absence of surface water reservoirs, it is useful to consider
streamflow regimes from the perspective of natural water storage, as an
intermediary between precipitation and stream flow at the outlet of the
watershed. At the end of the dry season, the watershed is in a
``draining from low storage'' condition. Snow storage has not been
available for several months, and groundwater storage is reaching its
annual low point. This is reflected in a slowly declining or flat
hydrograph, with a flowrate that has decreased for several months
(\autoref{fig:watershed_4states_illustration}, first period A). As the
dry season ends, the watershed begins receiving rain, and enters a
condition of ``filling from a low storage level''. In this catchment,
much of the earliest water entering the system is routed as recharge
through the soil or the streambed to occupy space in the aquifer.
Because groundwater moves more slowly through the watershed than surface
water, the hydrograph at the Fort Jones gauge demonstrates a muted or
delayed response to early rain events
(\autoref{fig:watershed_4states_illustration}, period B).

At the onset of a new wet season, under average conditions, the flowrate
of filling is greater than the flowrate of draining, and so the
``filling from a low storage level'' condition at the beginning of a
rainy season is transient, lasting only until the filling process
occupies enough aquifer and soil storage volume to produce a ``full''
condition. After the water storage in the basin reaches ``full'', if no
more rain occurs, the watershed returns to its default ``draining''
condition, though from a higher storage baseline than during the dry
season, and with a higher draining flowrate
(\autoref{fig:watershed_4states_illustration}, first period C). If there
is additional precipitation, the resulting surge in flow is much more
rapid, reflecting a ``spilling'' condition
(\autoref{fig:watershed_4states_illustration}, intermittent events D).

The precipitation and winter temperatures during the wet season produce
an accumulation of snowpack, though in some years this can be reduced by
warm periods and rain-on-snow events. Melting snowpack contributes
subsurface flow and tributary streamflow to the lower watershed,
producing a spring flow recession typically lasting from the last major
precipitation event into the summer
(\autoref{fig:watershed_4states_illustration}, second period C and
second period A) and providing significant recharge via stream leakage
near the mountain front, at the margins of the alluvial basin. This
process amounts to a net transfer of water in snow storage during winter
and spring to water in groundwater storage in spring and summer.

Many of these phenomena are are reflected in the well-established
elements of functional flows (\autoref{tab:watershed_modes_tab}). Winter
stormflow is the obvious functional flow metric corresponding to a
``spilling'' watershed. The spring recession can last for three to six
months and its steepness is moderated by snowmelt. Because it bridges
the high-storage and low-storage states, the early and late spring
recession appear in two different flow behavior categories
(\autoref{tab:watershed_modes_tab}). Conversely, the flows classified
under ``watershed filling from a low storage level'' are somewhat
ambiguous and dependent on year-to-year conditions, since a discrete
fall pulse flow does not occur in every water year, and no distinct
metric has been proposed for a more gradual post-dry-season flow
increase.

Given the regular behavior observed during the dry season-to-wet season
transition of the Fort Jones hydrograph (i.e., prolonged dry season
baseflows followed by gradual flow increase and then storm surges), and
the physical structure of this highly inter-connected basin, we expected
to find a flowrate threshold at the Fort Jones gauge approximately
defining the lower limit of the ``full'' or ``spilling'' basin condition
(referred to here as \(Q_{spill}\); see Results, Section 3.1).

\subsubsection{Stream-aquifer interaction}

In the groundwater basin portion of the watershed, the alluvial aquifer
is the largest storage reservoir \citep{Mack1958}. Groundwater-surface
water interactions drive Scott River flow behavior towards the end of
the dry season, before the next rainy season begins, when snowpack is
depleted and streamflow in many areas is sustained by groundwater
discharge alone \citep{Foglia2018a}. Discharge to streams from the
alluvial aquifer occurs along the thalweg of the Scott River. In this
highly interconnected system, groundwater discharge along the thalweg,
at the annual scale, is balanced by landscape recharge and by recharge
from stream reaches near the margins of the basin, overlying coarse
alluvial fans (see discussion below).

We used the Scott Valley Integrated Hydrologic Model (SVIHM)
\citep{Tolley2019, Foglia2013a, Foglia2013b} to obtain the estimated
volume of water exchanged monthly, in water years 1991-2018, between the
surface stream network and the underlying aquifer. Streamflow gains from
and streamflow losses to groundwater were integrated across the stream
network to obtain a net monthly groundwater-surface water exchange value
for the basin (\autoref{fig:flow_to_aq_and_stream}, panel A). These net
monthly groundwater-to-stream flux values were then compared to
simulated monthly flow volumes in the Scott River, measured at the Fort
Jones gauge (\autoref{fig:flow_to_aq_and_stream}, panel B).

\begin{figure}
\includegraphics[width=1\linewidth]{f03} \caption{\label{fig:flow_to_aq_and_stream} Both stream leakage and aquifer discharge increase in the rainy season, while net flux to the stream remains relatively close to 0 (panel A). Strong seasonal trends are evident in net flux to the stream (panel B; described further in text).}\label{fig:flow_to_aq_and_stream}
\end{figure}

\subsection{\texorpdfstring{Observed response variables (\(V_{min}\) and
\(P_{spill}\))}{Observed response variables (V\_\{min\} and P\_\{spill\})}}

The Scott River is an undammed watershed, in which estimates of annual
precipitation are an order of magnitude greater than the estimated water
pumped or diverted for agriculture \citep{Tolley2019}. In this study we
tested whether fundamental hydrologic characteristics, specifically the
dry-season baseflow recession and the rainfall-runoff response to early
wet season cumulative precipitation can be predicted five to six months
prior using observable hydroclimate data of the preceding wet season and
antecedent climate and flow conditions prior to the water year. The
first step was identification and quantification of relevant response
variables describing these two processes.

\subsubsection{\texorpdfstring{Dry season baseflow quantities
(\(V_{min}\)) and
timing}{Dry season baseflow quantities (V\_\{min\}) and timing}}

Multiple integrated numerical indicators of dry season baseflows were
evaluated for suitability as the response variables in this prediction
exercise. Monthly flow volumes were preferred over a minimum daily flow
value to reduce the influence of individual events that might affect
flow on one or a small number of days, such as groundwater pumping or
surface water diversions.

Historically, the rainy season in California tends to begin in October,
and so by convention each water year begins on October 1st of the
previous calendar year, and ends on September 30th. Matching this
convention, in most years, the minimum-flow month for the Scott River is
September; however, uncommon September storms can elevate flow volumes,
and in some years with a late rainy season onset, the October flow
volume may be lower. To capture these dynamics, for each calendar year,
we calculated a rolling 30-day sum of daily flow volumes in the period
July-December to identify the 30-day period with the least flow volume,
referred to as \(V_{min}\) (see Section 3.2). For consistency, each
annual \(V_{min}\) value was assigned to the water year ending in
September of that calendar year, even if the minimum flow window
included days in October-December of the following water year.

\subsubsection{\texorpdfstring{Cumulative precipitation
\(P_{spill}\)}{Cumulative precipitation P\_\{spill\}}}

\(P_{spill}\) was calculated for each water year as the cumulative
rainfall at the end of a dry season, between September 1st and the first
day that the Fort Jones gauge measured flow greater than \(Q_{spill}\)
(see Results, Section 3.2). As stated above, conceptually, it is the
amount of rainfall needed to ``fill'' the watershed such that streamflow
at the outlet of the watershed responds rapidly to new precipitation
without significant intervening storage delays. A dry season can have
negative effects on an aquatic ecosystem if it produces extraordinarily
low flows, but also if it lasts for an extraordinarily long time
\citeyearpar[e.g., delayed salmon habitat access, CDFW][]{CDFW2015a}.

In addition to the volumetric quantity \(P_{spill}\), there could also
be demand for seasonal predictions of the \emph{timing} of onset of the
coming rainy season. However, predicting the timing of the onset of the
rainy season or of \(Q_{spill}\) would likely rely on uncertain
long-term weather forecasts and is beyond the scope of this paper. In
other words, due to randomness in rainfall timing, the exact dry season
baseflow duration associated with a higher \(P_{spill}\) is highly
variable and, hence, unpredictable.

\subsection{Potential predictors and selected formulations}

To evaluate candidate predictors of dry season baseflows, Pearson's
correlation coefficient, \(R\), was calculated between observed response
variables \(V_{min}\) and \(P_{spill}\), and the following categories of
observed predictor data (see Results, Section 3.3):

\begin{enumerate}
\def\labelenumi{\arabic{enumi}.}
\tightlist
\item
  Spring (March-May) water level observations in this case study in each
  of 54 individual wells (\autoref{fig:gw_vs_fall_flows_corr_map}).
\item
  Annual maximum snowpack water content at each individual snow
  monitoring station here at 20 stations aggregated by the California
  Data Exchange Center (CDEC; \autoref{fig:watershed_fig_ch3}).
\item
  Cumulative precipitation, October-April, at local weather stations
  (here: 12 stations within and 5 outside the watershed for a total of
  17 NOAA stations; \autoref{fig:watershed_fig_ch3}). In these records,
  missing values (i.e., days with no recorded observation) are assumed
  to have 0 precipitation. Water years with more than 5 missing days are
  excluded from the predictor dataset.
\item
  Cumulative precipitation, October-April, of a composite precipitation
  record with no missing values, here representing the mean of the
  Callahan and Fort Jones NOAA weather stations (located at the southern
  and northern ends of the valley, respectively), and referred to as
  ``cal\_fj\_interp''. To generate the composite record, missing values
  in the Callahan and Fort Jones station records were imputed based on
  observations at neighboring stations \citep[see method
  in][]{Foglia2013a}.
\item
  The flow volumes observed at the watershed outlet, here the Fort Jones
  gauge (USGS ID 11519500; \autoref{fig:watershed_fig_ch3}) during the
  preceding March and April.
\item
  Cumulative reference evapotranspiration (ET\textsubscript{0}),
  October-April, here using observations from Station No.~225 in the
  California Information Management Information System (CIMIS) network
  (2015-2021), or Spatial CIMIS estimates of ET\textsubscript{0} at the
  location of Station 225 (2002-2015) (\autoref{fig:watershed_fig_ch3}).
\item
  The timing (in Julian days) of the date of maximum snowpack
  measurement.
\item
  The timing (in Julian days) of the date of the volumetric center of
  the rainy season, calculated as the day the cumulative precipitation
  crossed 50\% of the total.
\item
  The 1-year-lagged metrics of maximum snowpack, October-April
  cumulative precipitation, and April water levels (e.g., the
  October-April cumulative precipitation measured a full 17-23 months
  prior to a September minimum flow).
\end{enumerate}

Individual measuring locations, such as wells or weather stations, were
evaluated for sample size (i.e., years of data) and degree of
relatedness with the two response variables. Relatedness of the
monitoring locations with the highest \(R\) values in each category of
monitoring observation are included in analysis results (see Results,
Section 3.3).

\subsubsection{\texorpdfstring{Prediction formulae for \(V_{min}\) and
\(P_{spill}\)}{Prediction formulae for V\_\{min\} and P\_\{spill\}}}

With a sample size of 80 years of dry season baseflow volumes, a one- or
two-predictor model is best to avoid overfitting \citep{James2013}.
Nonetheless, to thoroughly assess predictive potential of this dataset,
three-predictor models were also considered.

Commonly, model diagnostics such as Akaike's Information Criterion (AIC)
are used to evaluate the best of a set of competing statistical models
\citep{Burnham2004}. However, AIC and other information criteria methods
can only be used when comparing models based on the same dataset
\citep{Burnham2004}. In the dataset for this study, variable record
lengths and missing data points produced a situation in which the sample
size is different for most of the combinations of predictors under
consideration. Consequently, though AIC and additional diagnostics were
calculated for all models (Tables \ref{tab:vmin_1_pred_tab} through
\ref{tab:pspill_tab_3pred}), cross-validation was used in this study as
the primary model selection technique.

To predict \(V_{min}\), a set of six one-predictor models were generated
using the observation location from each category with the highest
\(R\), and model fit was evaluated using Leave One Out Cross Validation
(LOOCV) \citep{James2013} (see Results, Section 3.3 and
\autoref{tab:vmin_1_pred_tab}). For a dataset with \(n\) observations,
the LOOCV error of a predictive model is obtained by recalculating the
model coefficients \(n\) times, each time leaving out one observation,
and comparing the resulting prediction to the single left-out
observation. The root mean square of these \(n\) errors is the LOOCV
error used to evaluate model performance in Results.

The single predictors with the lowest LOOCV error (other than
evapotranspiration of a reference crop, {[}\(ET_{ref}\){]}, which was
excluded due to insufficient observation record length) were used to
produce a set of four two-predictor models (see Results, Section 3.4 and
\autoref{tab:vmin_tab_2pred}) for \(V_{min}\), including two that
incorporate data from a full calendar year prior. A similar approach was
used to assess two-predictor models for \(P_{spill}\), though no
predictors from a prior year were included, and several additional
two-predictor combinations were evaluated. In both cases, the
best-performing model took the following form:

\begin{equation}
Predicted_{i} = Int. + m_A * obs_{A,~i}+m_B*obs_{B,~i}
\end{equation}

Where:

\begin{itemize}
\tightlist
\item
  \(Predicted_i\) is the predicted value (either \(V_{min}\) or
  \(P_{spill}\)) in calendar year \(i\) (i.e., at the end of water year
  \(i\)).
\item
  \(obs_{A,~i},~obs_{B,~i}\) are the observed predictor values in
  October-April in water year \(i\).
\item
  \(Int.,~m_A,~m_B\) are the coefficients of the selected linear model.
\end{itemize}

\section{Results}

\subsection{Scott River precipitation-runoff behavior}

The quantity \(P_{spill}\) is correlated with both a lower minimum flow
volume and a later river reconnection
(\autoref{fig:p_spill_vs_baseflow_and_recon_timing}). This corroborates
the hypothesis that \(P_{spill}\), predicted in advance, could be an
indicator of the risk of a severe dry season.

\begin{figure}
\includegraphics[width=1\linewidth]{f04} \caption{\label{fig:p_spill_vs_baseflow_and_recon_timing} The quantity P spill (i.e., the amount of rainfall needed to 'fill' the watershed such that it 'spills', or responds rapidly to new precipitation) is correlated with both a lower minimum dry season baseflow volume (panel A) and a later date of river reconnection (panel B).}\label{fig:p_spill_vs_baseflow_and_recon_timing}
\end{figure}

Visual inspection of 80 years of Fort Jones hydrograph behavior during
the transition from the dry season to the rainy season
(\autoref{fig:time_v_fall_rains_v_flow_fig}, panel A) indicated that
there were two distinct domains of flow: one in which flow is relatively
flat (dry season baseflow), and one in which the flowrate is an order of
magnitude higher, and it is highly responsive to rain events. The
intermediate hydrologic state, ``filling from low storage'', was visible
in some fall-winter hydrographs (e.g.,
\autoref{fig:watershed_4states_illustration}), but tended to last a
relatively short time before the filling rate overwhelmed the draining
rate and produced a responsive ``spilling'' condition.

The approximate threshold between these two hydrologic periods, denoted
as \(Q_{spill}\), was 120 cfs (294 thousand m\textsuperscript{3}/day or
approximately 9 Mm\textsuperscript{3} per month). This value was
determined by testing a range of potential \(Q_{spill}\) values to
evaluate rainfall-runoff responses preceding and following the threshold
(\autoref{fig:qspill_figs}).

\begin{figure}
\includegraphics[width=1\linewidth]{f05} \caption{\label{fig:time_v_fall_rains_v_flow_fig} In all three panels, 80 years of data series from September 1 to March 31 are overplotted to illustrate dynamics during the transition from the dry to the wet season: observed Fort Jones hydrographs in Panel A; cumulative rainfall and Fort Jones flow values on fall and winter days in Panel B; and cumulative rainfall over time in Panel C.}\label{fig:time_v_fall_rains_v_flow_fig}
\end{figure}

\subsection{Observed response variables}

\subsubsection{Dry season baseflow quantity and timing}

Minimum 30-day dry season baseflow volumes, denoted here as \(V_{min}\),
ranged from 0.3 to 7.5 Mm\textsuperscript{3} / 30 days, with one outlier
value of 13.9 Mm\textsuperscript{3} / 30 days in 1984, when an early
September storm followed a wet year in 1983
(\autoref{fig:fall_flows_data_exp}).

As mentioned previously, three eras have been proposed for the Scott
River flow record: Eras 1, 2, and 3, ranging from 1942-1976, 1977-2000,
and 2001-2021, respectively. Matching other long-term declining flow
trends in this watershed, the flows in August and September are
relatively steady in Era 1, and they become more variable with
significantly lower lows in Eras 2 and 3 (minima of 2.1, 0.35, and 0.33
Mm\textsuperscript{3} / 30 days {[}28.6, 4.8 and 4.4 average cfs{]}, in
Eras 1, 2 and 3 respectively; \autoref{fig:fall_flows_data_exp}, panel
A).

The timing of the midpoint of the 30-day minimum-flow period falls most
commonly in September, though it has fluctuated over the last eight
decades (\autoref{fig:fall_flows_data_exp}, panel B).

\begin{figure}
\includegraphics[width=1\linewidth]{f06} \caption{\label{fig:fall_flows_data_exp} Panel A: FJ Gauge flow volume, by year, aggregated to monthly time windows in the late summer, fall, and early winter. Eras are noted that correspond to various management and climate factors. Panel B: The timing of dry season minimum flows has ranged from late August to mid-October over the past 8 decades. Panel C: P spill has trended upward over the period of record.}\label{fig:fall_flows_data_exp}
\end{figure}

\subsubsection{Cumulative fall precipitation and watershed response}

Typically, flow at the Fort Jones gauge is low (i.e., \textless{} 100
cfs) and stable through the end of the dry season
(\autoref{fig:time_v_fall_rains_v_flow_fig}, panel A). However, in some
water years prior to the 1980s, the Fort Jones flowrate exceeded
\(Q_{spill}\) on September 1st
(\autoref{fig:time_v_fall_rains_v_flow_fig}, panels A and B), indicating
that even under persistent dry season draining conditions, under the
climate and water use conditions of wet years in the mid-20th century,
the Scott River remained responsive to new precipitation year-round. As
a result, the range of \(P_{spill}\), the cumulative precipitation
necessary to reach \(Q_{spill}\), is wide (0 to 178 mm, or 0 to 7
inches) (\autoref{fig:fall_flows_data_exp}, panel C). Mean \(P_{spill}\)
values were 57, 79, and 76 mm (2.3, 3.1, and 3 inches) in Eras 1, 2 and
3, respectively.

\subsection{Comparison with California DWR Water Year Type (WYT)
category}

\begin{figure}
\includegraphics[width=1\linewidth]{f07} \caption{\label{fig:resp_vars_wyt} DWR Water Year Type indices over time and compared to the two metrics of hydrologic conditions developed in this study: minimum 30-day dry season baseflow volume (V min) and the amount of precipitation necessary to produce 120 cfs flow in the Scott River (P spill).}\label{fig:resp_vars_wyt}
\end{figure}

The DWR water year type categories map fairly well onto the two proposed
hydrologic indices \(V_{min}\) and \(P_{spill}\)
(\autoref{fig:resp_vars_wyt}, panels A and B), which is to be expected,
as both DWR WYT and the two proposed indices are based in part on
cumulative precipitation data. However, there is less of an ability to
identify a long-term trend in the DWR WYT index time series than in the
time series of observed or predicted \(V_{min}\) or \(P_{spill}\)
values. Likely causes include the information lost when binning water
years into five categories, and the 30-year ranking window that would
prevent a direct comparison of post-2000 WYTs with pre-1950s WYTs
(\autoref{fig:resp_vars_wyt}, panel C).

\subsection{\texorpdfstring{Predictor comparison for \(V_{min}\) and
\(P_{spill}\)}{Predictor comparison for V\_\{min\} and P\_\{spill\}}}

\begin{figure}
\includegraphics[width=1\linewidth]{f08} \caption{\label{fig:corr_matrix} Correlation coefficient matrix of two response variables, minimum 30-day dry season baseflow volumes (V min) and cumulative precipitation necessary to produce 120 cfs in the Scott River (P spill), with various possible predictor metrics. Gray, purple, and blue squares highlight the inter-category correlation coefficients of snowpack metrics, Oct-April cumulative precipitation, and March-May groundwater elevation measurements. Red rectangles highlight the predictors with the greatest absolute correlation coefficient values with V min and P spill, respectively. Yellow rectangles highlight correlations with previous-year hydroclimate quantities.}\label{fig:corr_matrix}
\end{figure}

The observations of spring flows, snowpack, valley floor precipitation,
and groundwater elevation are positively correlated both within each
category and to each other overall, which is unsurprising: wet years are
associated with higher values in all of these categories. Groundwater
wells with highest predictive power tend to have long records (e.g.,
\(n\) of 10 or greater years) and to be close to the Scott River; these
results incorporate two wells proximate to the river, with record
lengths of 43 and 57 years (\autoref{fig:gw_vs_fall_flows_corr_map}).

Both response variables are correlated with four categories of
observations: spring flowrates, maximum snow water content, cumulative
precipitation recorded at weather stations on or near the valley floor
(October-April), and March-May water levels in some wells. Observations
in these categories are positively correlated with \(V_{min}\) and
negatively correlated with \(P_{spill}\). The correlation coefficient,
\(R\), of these response-predictor relationships range from 0.5 to 0.73
for \(V_{min}\) and from -0.38 to -0.66 for \(P_{spill}\)
(\autoref{fig:corr_matrix}).

In contrast to these four categories, October-April cumulative
ET\textsubscript{0} is negatively correlated with \(V_{min}\) and
positively correlated with \(P_{spill}\) (\(R\) of -0.68 and 0.48 for
\(V_{min}\) and \(P_{spill}\), respectively). October-April cumulative
ET\textsubscript{0} is also negatively correlated with snow,
precipitation, and groundwater level indicators. This could be because
ET can remove a significant volume of water from the watershed, or
because years with more rainy or stormy days accumulate less total
insolation and atmospheric water demand. Additionally, the relatively
high correlation (\(R\)) for between ET\textsubscript{0} and the two
response variables may be due to a small sample size, as all available
ET\textsubscript{0} observations or estimates were collected in 2002 or
later (i.e., in Era 3; \autoref{fig:one_predictor_model}).

Some meteorological timing was evaluated in addition to quantities. Both
response variables are moderately-to-weakly correlated with snow timing
(i.e., the Julian day of the maximum measured snowpack in a given year;
\(R\) of 0.33 to 0.52 and -0.22 to -0.38 for \(V_{min}\) and
\(P_{spill}\), respectively), but no significant correlation is evident
between the response variables and precipitation timing
(\autoref{fig:corr_matrix}).

Finally, a subset of observations from the previous water year were
included in the correlation matrix to test for multi-year influence on
the response variables. These previous-year metrics had very little
predictive power regarding minimum flows (correlated with \(V_{min}\)
with an \(R\) of 0.29 to 0.33) and virtually none regarding
\(P_{spill}\) (correlated with \(P_{spill}\) with an \(R\) of -0.01 to
-0.1).

\subsection{\texorpdfstring{Predicted values of
\(V_{min}\)}{Predicted values of V\_\{min\}}}

\subsubsection{\texorpdfstring{\(V_{min}\) predictor assessment and
prediction
formula}{V\_\{min\} predictor assessment and prediction formula}}

Six hydroclimate data categories contained at least one measurement
station with an \(R\) value greater than 0.5 between a data record and
observed \(V_{min}\) values. The single monitoring location in each of
these categories with the highest \(R\) value was selected for further
analysis (\autoref{fig:one_predictor_model}). (Although an \(R\) value
of 0.5 is not considered a strong correlation on its own, it was
selected as an inclusion threshold to retain any predictors that could
provide useful information in combination with other predictors.) Out of
this set of six hydroclimate data records, the maximum snowpack and
October-April cumulative precipitation produce the lowest model errors
(LOOCV errors of 2.74 and 2.72 Mm\textsuperscript{3}, respectively;
\autoref{fig:one_predictor_model}, top two panels;
\autoref{tab:vmin_1_pred_tab}). Among combinations of two predictors, a
linear combination of maximum snowpack and cumulative precipitation
improved on the best single-predictor model, with an LOOCV error of 2.29
Mm\textsuperscript{3} (\autoref{fig:two_predictor_model}, upper left
panel; \autoref{tab:vmin_tab_2pred}). No three-predictor models produced
lower LOOCV or higher \(R^2\) values (\autoref{tab:vmin_tab_3pred}), so
no visualizations were made of three-predictor model results.

Among the two-predictor models evaluated was a combination of maximum
snowpack water content and the timing of the maximum measurement
(\autoref{fig:two_predictor_model}, top right panel). This produced a
slightly larger error (2.78 Mm\textsuperscript{3}) than the
single-predictor model with maximum snowpack water content alone (2.74
Mm\textsuperscript{3}; \autoref{fig:one_predictor_model}, middle left
panel), indicating that the timing of maximum snow accumulation is
either relatively unimportant in generating dry season baseflows or that
the actual timing of snowpack maximum is not captured in temporally
sparse snow course measurements.

Additionally, two models featuring a partial one-year holdover were
evaluated, to test the validity of this component of the methodology of
DWR's Water Year Type index. In both cases, the addition of the climate
data from the previous year produced a very small change in model error
(\autoref{fig:two_predictor_model}, two lower panels), indicating that
in the Scott Valley context, the previous year's climate may have a
minor influence on dry season flows relative to the immediately
preceding rainy season.

Based on these results, the model selected to provide the best
\(V_{min}\) prediction formulation was a linear combination of snowpack
maximum from the Swampy John (SWJ) snow station (with data collected by
CDEC) and cumulative October-April precipitation from the Fort Jones
Ranger Station (FJRS) weather station (with data collected by NOAA) as
follows:

\begin{equation}
V_{min,~i} = -1.33 + 0.0053 * FJRS_{i}+0.0027*SWJ_{i}
\end{equation}

Where:

\begin{itemize}
\tightlist
\item
  \(V_{min,~i}\) is the predicted value of minimum 30-day dry season
  baseflows in calendar year \(i\) (i.e., at the end of water year
  \(i\)) (million m\textsuperscript{3} or Mm\textsuperscript{3})
\item
  \(SWJ_{i}\) is the maximum snow water content recorded at the Swampy
  John snow course (CDEC station ID SWJ or 285) in water year \(i\)
  (millimeters)
\item
  \(FJRS_{i}\) is the cumulative precipitation, recorded October-April
  of water year \(i\), measured at the Fort Jones Ranger Station (NOAA
  station ID USC00043182) (millimeters)
\end{itemize}

Diagnostics for the linear regression models assessed in the \(V_{min}\)
analysis are included in the Appendix (Figures
\ref{fig:one_predictor_model} through \ref{fig:v_min_over_time} and
Tables \ref{tab:vmin_1_pred_tab} through \ref{tab:vmin_tab_3pred}).

\subsubsection{\texorpdfstring{Predicted and observed \(V_{min}\) over
time}{Predicted and observed V\_\{min\} over time}}

The \(V_{min}\) formulation proposed above predicts minimum 30-day dry
season baseflows with a model error of 2.29 Mm\textsuperscript{3} per 30
days (31.2 cfs), and a root mean squared error of 1.4
Mm\textsuperscript{3} (19.4 cfs). This RMSE indicates substantial
uncertainty in any single year's prediction: it corresponds to 49\% of
the mean \(V_{min}\) value, 2.9 Mm\textsuperscript{3} (40 cfs).

Matching the historical trends of decreasing snowpack, the observed and
predicted \(V_{min}\) values show a downward trend over time
(\autoref{fig:v_min_over_time}, top panel). An outlier in the year 1984
reflects extremely high minimum dry season baseflows, relative to the
predicted values and the overall distribution. In that year, a
relatively high-baseflow season was followed by an early September
storm. The model residual (predicted minus observed flow volumes) for
1984 is also an outlier, indicating that the model has a sufficient
sample size to not be overwhelmed by this extreme value produced by an
exceedingly rare sequence of events (\autoref{fig:v_min_over_time},
middle panel).

The predictive \(V_{min}\) model is based on observations from the full
record, but three additional models were generated based on only the
observations from each period: Eras 1, 2, and 3, respectively. Residuals
based on Era 1 data are similar to those of the full record, with a
slight but systematic overprediction in Era 3; Era 2 residuals tend to
overpredict in Era 1 more than the full record; and Era 3 residuals
offer better performance in Era 3 than the full record, but produce
significant systematic underpredictions pre-2000
(\autoref{fig:v_min_over_time}, middle panel).

\subsection{\texorpdfstring{Predicted values of
\(P_{spill}\)}{Predicted values of P\_\{spill\}}}

\subsubsection{\texorpdfstring{\(P_{spill}\) predictor assessment and
prediction
formula}{P\_\{spill\} predictor assessment and prediction formula}}

The two single predictors of \(P_{spill}\) with the (largest \(R\) or
least error or both) were October-April cumulative precipitation and the
maximum snowpack (\autoref{fig:one_predictor_model_p_spill}, top two
panels), the same parameters that provided the best prediction
\(V_{min}\). The LOOCV model errors are 850 mm and 718 mm, respectively.
As in the \(V_{min}\) model development, \(ET_{0}\) was excluded from
consideration due to short record length. The best two-predictor model
was the combination of the two best single predictors, with an LOOCV
error of 697 mm (\autoref{fig:two_predictor_model_p_spill}, upper left
panel). No three-predictor models produced lower LOOCV or \(R^2\) values
(\autoref{tab:pspill_tab_3pred}), so no visualizations were made of
three-predictor model results.

Several combinations of correlated observation categories produced
comparable model results, such as spring water levels with maximum snow,
maximum snow timing, and cumulative precipitation
(\autoref{fig:two_predictor_model_p_spill}, upper right and two middle
panels). However, not all combinations of co-correlated data produced
reasonable predictors; a model with a linear combination of maximum
snowpack timing and March flow volumes performed relatively poorly
(LOOCV error of 1,087 mm; \autoref{fig:two_predictor_model_p_spill},
lower right panel).

Based on these results, the model selected as the \(P_{spill}\)
formulation for a given water year was a linear combination of the same
observation records as \(V_{min}\): snowpack maximum from the SWJ snow
station (with data collected by CDEC) and cumulative October-April
precipitation from the Fort Jones Ranger Station weather station
(station ID USC00043182, with data collected by NOAA).

\begin{equation}
P_{spill,~i} = 128 -0.095 * FJRS_{i} - 0.028* SWJ_{i}
\end{equation}

Where:

\begin{itemize}
\tightlist
\item
  \(P_{spill,~i}\) is the predicted value of cumulative rainfall at the
  end of the dry season, starting Sep.~1, on the first day that the Fort
  Jones gauge records flow greater than or equal to 120 cfs in calendar
  year \(i\) (i.e., at the end of water year \(i\)) (millimeters)
\item
  \(SWJ_{i}\) is the maximum snow water content recorded at the Swampy
  John snow course (CDEC station ID SWJ or 285) in water year \(i\)
  (millimeters)
\item
  \(FJRS_{i}\) is the cumulative precipitation, recorded October-April
  of water year \(i\), measured at the Fort Jones Ranger Station (NOAA
  station ID USC00043182) (millimeters)
\end{itemize}

Diagnostics for the linear regression models assessed in thu
\(P_{spill}\) analysis are included in the Appendix (Figures
\ref{fig:one_predictor_model_p_spill} through
\ref{fig:pspill_pred_over_time} and Tables \ref{tab:pspill_tab_1pred}
through \ref{tab:pspill_tab_3pred}).

\subsubsection{\texorpdfstring{Predicted and observed \(P_{spill}\) over
time}{Predicted and observed P\_\{spill\} over time}}

The \(P_{spill}\) estimate formulation proposed above predicts
\(P_{spill}\) values with a model LOOCV error of 697 mm (27.4 inches),
and a root mean squared error of 25.4 mm (1 inch). This RMSE indicates
substantial uncertainty in any single year's prediction: it corresponds
to 37\% of the mean \(P_{spill}\) value, 68.8 mm.

Matching the historical trends of decreasing snowpack, the observed and
predicted \(P_{spill}\) values show an increasing trend over time
(\autoref{fig:pspill_pred_over_time}, top panel). A high outlier in
calendar year 1994 (in early water year 1995) was caused by a dry water
year 1994 followed by a series of small storms in November and December,
none of which produced 120 cfs of flow, followed by a much larger storm
on January 8th-9th of 1995 in which the river flow jumped to 600 and
then 7,500 cfs in two days.

The predictive \(P_{spill}\) model is based on observations from the
full record, but three additional models were generated based on only
the observations from each period: Eras 1, 2, and 3, respectively.
Residuals based on Era 1 tend to underpredict Eras 2 and 3 more than the
full-record model; Era 2 residuals tend to overpredict in Eras 1 and 3
more than the full record; and Era 3 residuals have a slightly higher
tendency to underpredict than the full record, but overall are fairly
similar to the full-record residuals (\autoref{fig:v_min_over_time},
lower panel).

The linear coefficients for the two prediction equations and their
standard errors are summarized in \autoref{tab:coeff_and_std_error_tab}.

\begin{table}[ht]
\centering
\caption{Summary of linear model coefficients for V min and P spill.} 
\label{tab:coeff_and_std_error_tab}
\begin{tabular}{rrrr}
  \hline
 & b & m\_SWJ & m\_FJRS \\ 
  \hline
V min prediction & -1.3 & 0.0027 & 0.0053 \\ 
  V min std. error & 0.4 & 0.0006 & 0.0013 \\ 
  P spill prediction & 128.1 & -0.0947 & -0.0283 \\ 
  P spill std. error & 7.9 & 0.0228 & 0.0103 \\ 
   \hline
\end{tabular}
\end{table}

\section{Discussion}

\subsection{Scott River watershed behavior and long-term planning}

The forward-looking seasonal predictions in this study are possible only
because of predictable hydrologic relationships between early-season
inputs (precipitation) and late-season outputs (surface flow and ET). In
this watershed there are reasonably consistent rainfall-runoff
relationships: the general shape of the relationship between cumulative
precipitation and river flow at onset of the rainy season is preserved
in dry and wet water years (\autoref{fig:time_v_fall_rains_v_flow_fig},
panel B). This consistency is also reflected in stormflow behavior
occurring after precipitation fills the watershed, a condition in which
the Scott River exceeds a flowrate of \(Q_{spill}\) at the Fort Jones
stream gauge.

A \(Q_{spill}\) threshold of 120 cfs was identified by an analysis of
rainfall-runoff responses and a visual inspection of fall hydrograph
behavior (Figures \ref{fig:qspill_figs} and
\ref{fig:time_v_fall_rains_v_flow_fig}), and it also matches information
from local stakeholders. Many tributary streams on the valley floor run
dry during the summer and fall, and some tributary streams respond more
quickly to fall precipitation than others. Generally, the timing of all
tributaries reaching flowing status corresponds with the Fort Jones
gauge reaching 100 cfs \citep{Sommarstrom2020}.

Simulated estimates of stream-aquifer exchange corroborate these
precipitation-flow relationships. Dry season baseflow (\(V_{min}\)) and
the onset of wet season flow (framed in terms of \(P_{spill}\)) are both
influenced by net groundwater discharge from the aquifer. One
interpretation of the high frequency of near-0 net monthly
stream-aquifer flux values (\autoref{fig:flow_to_aq_and_stream}, panel
B) is that the high degree of connectivity between the streams and the
aquifer in the Scott River system produces balancing counter-forces in
response to hydrologic stresses on the system, such as large recharge
events. This balancing tendency can be temporarily overwhelmed by large
precipitation pulses, but high-flow conditions quickly reduce the volume
of water in the surface water system, returning the Scott River to a
baseline of relatively-balanced stream-to-aquifer and aquifer-to-stream
fluxes. This dynamic also reflects the small size of the available
aquifer storage, relative to the amount of precipitation received by the
watershed in a given water year \citeyearpar[DWR][]{DWR2004}, leading to
net groundwater contributions that are one to two orders of magnitude
smaller than streamflow during ``spilling'' conditions
(\autoref{fig:flow_to_aq_and_stream}, panel B) and two or more orders of
magnitude difference between dry season low flows (almost exclusively
from groundwater contributions) and wet season high flows.

These limitations in natural water storage preclude the use of wet year
surplus for dry years and dictate that water management focus on
within-year projects and management actions that either carry over some
of the wet season flow to enhance dry season flow (\(V_{min}\)) and
possibly reduce \(P_{spill}\) via interim aquifer storage, or reduce
surface water diversions and groundwater pumping, or a combination of
both. These conditions also reduce resiliency to climate change.

\subsection{\texorpdfstring{\(V_{min}\) and \(P_{spill}\) prediction
utility}{V\_\{min\} and P\_\{spill\} prediction utility}}

Though various methods exist to qualitatively predict, in the spring,
the severity of the coming low-flow season in the Scott River watershed,
a quantitative short-term forecasting index could support more rigorous
thresholds for adaptive management. To this end we developed two linear
equations for predicting minimum dry season baseflows about five months
in advance, effectively taking an inventory in each April of relevant
hydrologic inputs. It could be used to support decisions made in the
late spring timeframe regarding the growing season, such as potential
regulatory actions and some farmer cropping decisions.

There are several methods in current use. Observations at existing
monitoring locations, such as weather stations and long-term snow course
records, have been used as ad-hoc hydrologic indices. Historical
adaptive management decisions in the Scott River watershed, such as
planning to purchase surface water rights leases, have relied on
individual monitoring observations, such as percent of snowpack relative
to average conditions, or the Fort Jones flow in the spring
\citeyearpar[e.g., SRWT][]{SRWT2018}. Additionally, DWR has effectively
extended the methodology of the SVI and SJI metrics to all of California
by publishing a categorical water year type (WYT) index for all its
major watersheds (to the HUC8 level) \citeyearpar[DWR][]{DWR2021a}. This
metric quantifies meteoric drought and relies only on precipitation
data, so as to be comparable across the state. Matching SVI and SJI
methodology, it can be calculated at multiple points in each spring,
with a final determination in May, but in the case of Scott Valley it
has been used to classify WYTs only retroactively through 2018. As
previously mentioned it is a relatively complex metric with provisions
including a partial one-year holdover (i.e., dry conditions in the
previous year will make a dry-type categorization more likely the
following year), and non-stationary index thresholds, with a 30-year
ranking window.

The quantitative prediction methods proposed in this study map well onto
the existing DWR WYT index (\autoref{fig:resp_vars_wyt}), which serves
as a validation of this approach. However, the \(V_{min}\) and
\(P_{spill}\) metrics preserve more detail. The primary advantages of
the proposed method over the DWR WYT and other previous methods of
gauging near-term hydrologic conditions is that it is tailored to local
hydroclimate data and is interpretable as a numeric prediction of fall
conditions. This could be used to inform regulatory actions in an
attempt to increase fall environmental flows, or for surface water
diverters to plan for low-flow conditions. However, each seasonal
prediction should be accompanied by its uncertainty: the RMSE of this
predictive model is 49\% of the mean value for \(V_{min}\), 2.9
Mm\textsuperscript{3} (40 cfs).

Though it also serves as an indication of the severity of a water year,
the immediate seasonal utility of the second predicted metric,
\(P_{spill}\), may be less than for that of minimum dry season
baseflows. Management decisions such as the last possible date to keep a
temporary stream gauge installed in a river, without risk of it being
washed out, could be informed by a \(P_{spill}\) prediction when
combined with weather forecasts in the fall. This prediction also
carries substantial uncertainty (RMSE of 37\% of the mean \(P_{spill}\)
value, 68.8 mm).

\subsection{Management implications of best-performing predictors}

As described in Results, the linear models that best predicted observed
values of \(V_{min}\) and \(P_{spill}\) were both based on the same two
observation locations (the SWJ snow course and the FJRS weather station;
\autoref{fig:watershed_fig_ch3}), both with lengthy observation records.
One interpretation of these results is that the climate inputs produce a
predictable fall watershed response, and that human management decisions
have a negligible influence on fall river flow. However, model
simulations suggest that the timing and magnitude of fall flow increases
can be meaningfully influenced by human water use (e.g., scenarios in
the Siskiyou County GSP) \citeyearpar[Siskiyou
County][]{SiskiyouCounty2021}.

Multiple possible explanations could reconcile these two pieces of
seemingly contradictory evidence. First, random variability in human
water use could be a contributing factor to the error of the predictive
models of fall-season hydrologic behavior. Alternatively, human water
uses could be so consistent in response to wet or dry season conditions
that these water uses could be implicitly incorporated into the
predictive models. If adaptive management actions (potentially including
events as diverse as regulatory curtailments or individual cropping
choices) are carefully recorded in the future, they could be compared to
residuals of the climate-based predictive models to evaluate whether any
signal of a response to human interventions can be observed.

Potentially, these seasonal predictions could be extended to additional
watersheds (albeit ones with abundant available hydroclimate data).
However, the feasibility of this geographic generalization is beyond the
scope of this study and should be investigated in future work.

\subsection{Influences of climate change on predictive indices}

Both predictions (using the full record of hydrologic data) assume some
degree of hydroclimate stationarity, in that they use historical
snowpack- and precipitation-runoff relationships to predict modern
runoff. In one sense, a longer-term record can be an asset, in that it
provides context for the severity of the dry periods of the past two
decades. In another sense it is a liability for prediction accuracy: for
example, the predicted \(V_{min}\) values based on the full record
appear to systematically overpredict \(V_{min}\) in the most recent era
(2001-2021; \autoref{fig:two_predictor_model}, top left panel, and
\autoref{fig:v_min_over_time}, middle panel). This suggests that factors
not captured in these climate data, such as warmer air temperatures,
changing upland vegetation and evapotranspiration dynamics, and possibly
unknown changes in water use, may be altering the relationship between
the spring water supply and dry season baseflow rates. However, even
with detailed records of water use and management actions, disentangling
the influences of hydroclimate and human management on streamflow
conditions remains a complex challenge.

\section{Conclusions}

This study proposed two locally-tailored hydrologic decision-support
metrics for the Scott River watershed in northern California. Both use
snowpack and cumulative precipitation data from October-April to predict
the quantity of interest: the first is the minimum 30-day flow volume in
a given water year, referred to as \(V_{min}\), which typically occurs
in September or October. The second index is the cumulative rainfall
needed to ``fill'' the watershed after the end of the dry season to a
``spilling'' condition that responds quickly to precipitation events,
referred to as \(P_{spill}\). Both indices can be calculated at the end
of April to support near-term (seasonal) adaptive management regarding
the growing season or the fall, similar to the SVI and SJI in
California's Central Valley and other indices used throughout the
western United States. However, climate change may reduce the predictive
accuracy of indices based on long-term data records, and updates based
on smaller numbers of more recent water years should be considered
periodically.

The management choices facing local managers in this basin are difficult
to quantify and summarize, as is the case in basins throughout
California and arid regions globally. Locally-derived summary metrics,
tailored to regional hydrologic dynamics, have provided and will
continue to provide tools for supporting those choices and communicating
them to diverse stakeholders and the general public.

\newpage

\section{Appendix}

\appendixfigures
\appendixtables

\subsection{Groundwater Well Data}

Groundwater elevation measurements collected in the spring were
evaluated as a potential predictor of river flow behavior the following
fall. The length of the records and number of spring-season measurements
were variable, limiting the number of wells that could be used as
predictor sites. Two wells were selected for the final predictor
evaluation.

\begin{figure}
\includegraphics[width=1\linewidth]{f09} \caption{\label{fig:gw_vs_fall_flows_corr_map} Boundary of the groundwater basin (corresponding approximately to the extent of the flat valley floor in the Scott River watershed) and selected well locations. Colors correspond to the correlation coefficients between April groundwater elevations and September flow volume. The wells included in the predictor comparison are highlighted with a red outer square. 54 wells had enough spring-season water level measurements to be included in this correlation exercise, though some wells are so close together that their symbols overplot on this map.}\label{fig:gw_vs_fall_flows_corr_map}
\end{figure}

\newpage

\subsection{\texorpdfstring{Selection of the \(Q_{spill}\) threshold
value}{Selection of the Q\_\{spill\} threshold value}}

The \(Q_{spill}\) value of 120 cfs (294 thousand
m\textsuperscript{3}/day or approximately 9 Mm\textsuperscript{3} per
month) was determined by testing a range of potential \(Q_{spill}\)
values (10 to 500 cfs {[}24 to 1223 thousand
m\textsuperscript{3}/day{]}) as a dividing threshold between dry- and
wet-season flow behavior, and calculating the difference in
rainfall-runoff response on either side of the dry season-wet season
divide (\autoref{fig:qspill_figs}, Panel A). The objective of this
analysis was to maximize the difference between these two \(dQ/dP\)
values: to identify the threshold that reflected a maximally different
rainfall-runoff response before and after the threshold was reached.

The difference between the wet and dry season rainfall-runoff responses
is approximately the same (113 cfs {[}275 m\textsuperscript{3}/day{]} of
increased flow per mm of precipitation) for spill thresholds from
105-135 cfs (256-330 10\textsuperscript{3} m\textsuperscript{3}/day)
(\autoref{fig:qspill_figs}, Panel A). This indicates that in a large
number of water years, flows in the range of 105-135 cfs (257 to 330
thousand m\textsuperscript{3}/day) range are preceded by a dry season
flow-response regime and followed by a distinct, flashier flow regime
regime. Though higher wet-dry flow response differences were calculated
at higher threshold values (i.e., up to 500 cfs {[}1,223 thousand
m\textsuperscript{3}/day{]}), these progressively higher wet-season flow
responses likely reflect the falling limb of individual large storms
that over-fill the watershed rather than separating filling from
spilling behavior.

Additionally, in visual inspection of 78 years of Fort Jones
hydrographs, the 120 cfs (294 10\textsuperscript{3}
m\textsuperscript{3}/day ) \(Q_{spill}\) threshold generated a plausible
breakpoint between dry and wet season river behavior in all water years
(e.g., water year 2018 in \autoref{fig:qspill_figs}, Panel B).
Furthermore, this value corroborates local observations that an
approximate value of 100 cfs represents ``fully connected'' river
conditions (see Discussion section).

\begin{figure}
\includegraphics[width=1\linewidth]{f10} \caption{\label{fig:qspill_figs} This analysis (Panel A) aimed to identify the flow threshold that approximately defines the boundary between filling (i.e. dry season) and spilling (i.e. wet season) flow behavior at the Fort Jones gauge. For each threshold value, for each water year, a rainfall-runoff response was calculated before and after the flow threshold. The rainfall-runoff response consisted of the 30-day cumulative flow difference (dQ) per 30-day cumulative rainfall difference (dP). 120 cfs was selected as the threshold value dividing the dry and wet seasons (e.g., Panel B).}\label{fig:qspill_figs}
\end{figure}

\newpage

\subsection{\texorpdfstring{Model selection criteria -
\(V_{min}\)}{Model selection criteria - V\_\{min\}}}

Diagnostics used to select the predictive models for \(V_{min}\) are
shown below and discussed in Results. Predictors are abbreviated in
tables and described briefly in \autoref{tab:vmin_1_pred_tab}; for more
information on potential predictors see Section 2.3.

\begin{table}[ht]
\centering
\caption{Linear model diagnostics for one-predictor models of minimum fall flows (V min).} 
\label{tab:vmin_1_pred_tab}
\begin{tabular}{llrrrrr}
  \hline
Predictor ID & Predictor Descrip. & n & Log Likelihood & AIC & LOOCV & R squared \\ 
  \hline
SWJ\_max\_wc\_mm & Snow maximum & 70 & -131 & 269 & 2.7 & 0.53 \\ 
  USC00043182\_oct\_apr\_mm & Oct.-Apr. Precip. & 75 & -142 & 290 & 2.7 & 0.49 \\ 
  SWJ\_jday\_of\_max & Snow maximum timing & 70 & -154 & 314 & 5.1 & 0.11 \\ 
  springWL\_415635N1228315W001 & March-May WLs & 50 & -100 & 206 & 3.6 & 0.39 \\ 
  et0\_oct\_apr & ET Ref. & 17 & -21 & 48 & 0.9 & 0.46 \\ 
  mar\_flow & March flow vol. & 78 & -162 & 330 & 4.1 & 0.25 \\ 
   \hline
\end{tabular}
\end{table}

\begin{figure}
\includegraphics[width=1\linewidth]{f11} \caption{\label{fig:one_predictor_model} Single-predictor models of minimum 30-day dry season baseflows in the Scott River.}\label{fig:one_predictor_model}
\end{figure}

\begin{table}[ht]
\centering
\caption{Linear model diagnostics for two-predictor models of V min. See table of one-predictor models for description of predictor IDs. Reference ET was not included in two- and three-predictor models due to an insufficient sample size.} 
\label{tab:vmin_tab_2pred}
\begin{tabular}{llrrrrr}
  \hline
Predictor 1 & Predictor 2 & n & Log Likelihood & AIC & LOOCV & R squared \\ 
  \hline
SWJ\_max\_wc\_mm & USC00043182\_oct\_apr\_mm & 67 & -119 & 246 & 2.3 & 0.62 \\ 
  SWJ\_max\_wc\_mm & SWJ\_jday\_of\_max & 70 & -131 & 270 & 2.8 & 0.53 \\ 
  SWJ\_max\_wc\_mm & springWL\_415635N1228315W001 & 50 & -91 & 191 & 2.7 & 0.57 \\ 
  SWJ\_max\_wc\_mm & mar\_flow & 70 & -128 & 264 & 2.6 & 0.57 \\ 
  USC00043182\_oct\_apr\_mm & SWJ\_jday\_of\_max & 67 & -127 & 263 & 2.9 & 0.52 \\ 
  USC00043182\_oct\_apr\_mm & springWL\_415635N1228315W001 & 47 & -91 & 190 & 3.3 & 0.48 \\ 
  USC00043182\_oct\_apr\_mm & mar\_flow & 75 & -139 & 286 & 2.8 & 0.52 \\ 
  SWJ\_jday\_of\_max & springWL\_415635N1228315W001 & 50 & -97 & 203 & 3.3 & 0.45 \\ 
  SWJ\_jday\_of\_max & mar\_flow & 70 & -141 & 291 & 3.8 & 0.37 \\ 
  springWL\_415635N1228315W001 & mar\_flow & 50 & -94 & 197 & 3.1 & 0.51 \\ 
   \hline
\end{tabular}
\end{table}

\begin{table}[ht]
\centering
\caption{Linear model diagnostics for three-predictor models of minimum fall flows (V min). See table of one-predictor models for description of predictor IDs. Reference ET was not included in two- and three-predictor models due to an insufficient sample size.} 
\label{tab:vmin_tab_3pred}
\begingroup\fontsize{8pt}{9pt}\selectfont
\begin{tabular}{lllrrrrr}
  \hline
Predictor 1 & Predictor 2 & Predictor 3 & n & Log Like. & AIC & LOOCV & R squared \\ 
  \hline
SWJ\_max\_wc\_mm & USC00043182\_oct\_apr\_mm & SWJ\_jday\_of\_max & 67 & -119 & 247 & 2.3 & 0.63 \\ 
  SWJ\_max\_wc\_mm & USC00043182\_oct\_apr\_mm & springWL\_415635N1228315W001 & 47 & -86 & 181 & 2.8 & 0.59 \\ 
  SWJ\_max\_wc\_mm & USC00043182\_oct\_apr\_mm & mar\_flow & 67 & -117 & 244 & 2.3 & 0.64 \\ 
  SWJ\_max\_wc\_mm & SWJ\_jday\_of\_max & springWL\_415635N1228315W001 & 50 & -91 & 192 & 2.7 & 0.58 \\ 
  SWJ\_max\_wc\_mm & SWJ\_jday\_of\_max & mar\_flow & 70 & -127 & 265 & 2.7 & 0.58 \\ 
  SWJ\_max\_wc\_mm & springWL\_415635N1228315W001 & mar\_flow & 50 & -89 & 187 & 2.6 & 0.61 \\ 
  USC00043182\_oct\_apr\_mm & SWJ\_jday\_of\_max & springWL\_415635N1228315W001 & 47 & -90 & 190 & 3.3 & 0.51 \\ 
  USC00043182\_oct\_apr\_mm & SWJ\_jday\_of\_max & mar\_flow & 67 & -123 & 256 & 2.8 & 0.58 \\ 
  USC00043182\_oct\_apr\_mm & springWL\_415635N1228315W001 & mar\_flow & 47 & -87 & 184 & 3.0 & 0.56 \\ 
  SWJ\_jday\_of\_max & springWL\_415635N1228315W001 & mar\_flow & 50 & -91 & 191 & 2.9 & 0.58 \\ 
   \hline
\end{tabular}
\endgroup
\end{table}

\begin{figure}
\includegraphics[width=1\linewidth]{f12} \caption{\label{fig:two_predictor_model} Two-predictor models of minimum 30-day dry season baseflows in the Scott River.}\label{fig:two_predictor_model}
\end{figure}

\begin{figure}
\includegraphics[width=1\linewidth]{f13} \caption{\label{fig:v_min_over_time} Observed and predicted minimum 30-day dry season baseflows both trend downward between the three eras of the period of record (top panel). The predicted-minus-observed difference (residual) over time also reflects this trend, underpredicting minimum flows pre-1977 and overpredicting them post-2000 (middle panel). The predictive model is based on observations from the full record, but three additional models were generated based on only the observations from Eras 1, 2, and 3. Residuals based on Era 1 data are similar to those of the full record; Era 2 residuals tend to overpredict more than the full record; and Era 3 residuals show better performance post-2000 than the full record, but significant underprediction pre-2000.}\label{fig:v_min_over_time}
\end{figure}

\newpage

\subsection{\texorpdfstring{Model selection criteria -
\(P_{spill}\)}{Model selection criteria - P\_\{spill\}}}

Diagnostics used to select the predictive models for \(P_{spill}\) are
shown below and discussed in Results. Predictors are abbreviated in
tables and described briefly in \autoref{tab:pspill_tab_1pred}; for more
information on potential predictors see Section 2.3.

\begin{table}[ht]
\centering
\caption{Linear model diagnostics for one-predictor models of P spill.} 
\label{tab:pspill_tab_1pred}
\begin{tabular}{llrrrrr}
  \hline
Predictor ID & Predictor Descrip. & n & Log Likelihood & AIC & LOOCV & R squared \\ 
  \hline
SWJ\_max\_wc\_mm & Snow maximum & 70 & -333 & 673 & 850 & 0.38 \\ 
  USC00043182\_oct\_apr\_mm & Oct.-Apr. Precip. & 75 & -351 & 708 & 718 & 0.43 \\ 
  SWJ\_jday\_of\_max & Snow maximum timing & 70 & -347 & 699 & 1243 & 0.09 \\ 
  springWL\_415635N1228315W001 & March-May WLs & 50 & -245 & 495 & 1123 & 0.24 \\ 
  et0\_oct\_apr & ET Ref. & 17 & -81 & 167 & 932 & 0.23 \\ 
  mar\_flow & March flow vol. & 78 & -380 & 767 & 1061 & 0.14 \\ 
   \hline
\end{tabular}
\end{table}

\begin{table}[ht]
\centering
\caption{Linear model diagnostics for two-predictor models of P spill. See table of one-predictor models for description of predictor IDs. Reference ET was not included in two- and three-predictor models due to an insufficient sample size.} 
\label{tab:pspill_tab_2pred}
\begin{tabular}{llrrrrr}
  \hline
Predictor 1 & Predictor 2 & n & Log Likelihood & AIC & LOOCV & R squared \\ 
  \hline
SWJ\_max\_wc\_mm & USC00043182\_oct\_apr\_mm & 67 & -312 & 631 & 697 & 0.51 \\ 
  SWJ\_max\_wc\_mm & SWJ\_jday\_of\_max & 70 & -333 & 674 & 870 & 0.39 \\ 
  SWJ\_max\_wc\_mm & springWL\_415635N1228315W001 & 50 & -240 & 487 & 951 & 0.38 \\ 
  SWJ\_max\_wc\_mm & mar\_flow & 70 & -333 & 674 & 853 & 0.39 \\ 
  USC00043182\_oct\_apr\_mm & SWJ\_jday\_of\_max & 67 & -315 & 637 & 760 & 0.47 \\ 
  USC00043182\_oct\_apr\_mm & springWL\_415635N1228315W001 & 47 & -224 & 457 & 920 & 0.43 \\ 
  USC00043182\_oct\_apr\_mm & mar\_flow & 75 & -351 & 709 & 728 & 0.44 \\ 
  SWJ\_jday\_of\_max & springWL\_415635N1228315W001 & 50 & -243 & 493 & 1068 & 0.30 \\ 
  SWJ\_jday\_of\_max & mar\_flow & 70 & -341 & 690 & 1079 & 0.23 \\ 
  springWL\_415635N1228315W001 & mar\_flow & 50 & -242 & 493 & 1067 & 0.31 \\ 
   \hline
\end{tabular}
\end{table}

\begin{table}[ht]
\centering
\caption{Linear model diagnostics for three-predictor models of P spill. See table of one-predictor models for description of predictor IDs. Reference ET was not included in two- and three-predictor models due to an insufficient sample size.} 
\label{tab:pspill_tab_3pred}
\begingroup\fontsize{8pt}{9pt}\selectfont
\begin{tabular}{lllrrrrr}
  \hline
Predictor 1 & Predictor 2 & Predictor 3 & n & Log Like. & AIC & LOOCV & R squared \\ 
  \hline
SWJ\_max\_wc\_mm & USC00043182\_oct\_apr\_mm & SWJ\_jday\_of\_max & 67 & -311 & 633 & 712 & 0.52 \\ 
  SWJ\_max\_wc\_mm & USC00043182\_oct\_apr\_mm & springWL\_415635N1228315W001 & 47 & -222 & 455 & 874 & 0.47 \\ 
  SWJ\_max\_wc\_mm & USC00043182\_oct\_apr\_mm & mar\_flow & 67 & -312 & 633 & 714 & 0.51 \\ 
  SWJ\_max\_wc\_mm & SWJ\_jday\_of\_max & springWL\_415635N1228315W001 & 50 & -239 & 488 & 973 & 0.39 \\ 
  SWJ\_max\_wc\_mm & SWJ\_jday\_of\_max & mar\_flow & 70 & -332 & 675 & 872 & 0.40 \\ 
  SWJ\_max\_wc\_mm & springWL\_415635N1228315W001 & mar\_flow & 50 & -239 & 488 & 955 & 0.40 \\ 
  USC00043182\_oct\_apr\_mm & SWJ\_jday\_of\_max & springWL\_415635N1228315W001 & 47 & -224 & 457 & 936 & 0.44 \\ 
  USC00043182\_oct\_apr\_mm & SWJ\_jday\_of\_max & mar\_flow & 67 & -314 & 638 & 768 & 0.48 \\ 
  USC00043182\_oct\_apr\_mm & springWL\_415635N1228315W001 & mar\_flow & 47 & -223 & 457 & 923 & 0.45 \\ 
  SWJ\_jday\_of\_max & springWL\_415635N1228315W001 & mar\_flow & 50 & -240 & 490 & 1008 & 0.37 \\ 
   \hline
\end{tabular}
\endgroup
\end{table}

\begin{figure}
\includegraphics[width=1\linewidth]{f14} \caption{\label{fig:one_predictor_model_p_spill} Single-predictor models of P spill, the cumulative precipitation after the dry season needed to generate 120 cfs of flow in the Scott River.}\label{fig:one_predictor_model_p_spill}
\end{figure}

\begin{figure}
\includegraphics[width=1\linewidth]{f15} \caption{\label{fig:two_predictor_model_p_spill} Two-predictor models of P spill, the cumulative precipitation after the dry season needed to generate 120 cfs of flow in the Scott River.}\label{fig:two_predictor_model_p_spill}
\end{figure}

\begin{figure}
\includegraphics[width=1\linewidth]{f16} \caption{\label{fig:pspill_pred_over_time} Observed and predicted values of P spill (panel A) indicate a worse model fit for the P spill prediction than for minimum 30-day dry season baseflows. Serious overprediction in Era 1 is followed by more mixed over- and under-prediction in Eras 2 and 3 (bottom panel). The overall P spill model is based on observations from the full record, but three additional models were generated based on only the observations from Eras 1, 2, and 3. Residuals based on Era 1 data are generally lower than those from Eras 2 or 3 or from the full record.}\label{fig:pspill_pred_over_time}
\end{figure}

\newpage

\subsection{Diagnostic plots for selected models}

Standard diagnostic plots for the selected predictive models for
\(V_{min}\) and \(P_{spill}\). For \(V_{min}\), these diagnostic plots
highlight outlier record 42, which corresponds to the year 1983, when an
early September storm followed a wet year. For \(P_{spill}\), three
lesser outliers are highlighted, corresponding to water years 1994, 2000
and 2004, in which the three highest values of \(P_{spill}\) were
observed. These outliers represent the basin in extreme hydrologic
conditions, so are retained in the dataset even though they exert
disproportionate leverage over the predictive models.

\begin{figure}
\includegraphics[width=1\linewidth]{Kouba_2023_end_of_dry_season_manuscript_Comment-Responses-due-2023.07.24_files/figure-latex/selected_model_diagnostics_vmin-1} \caption{\label{fig:vmin_selected_diagnostics} Diagnostic plots for the selected predictive models of V min.}\label{fig:selected_model_diagnostics_vmin}
\end{figure}

\begin{figure}
\includegraphics[width=1\linewidth]{Kouba_2023_end_of_dry_season_manuscript_Comment-Responses-due-2023.07.24_files/figure-latex/selected_model_diagnostics_pspsill-1} \caption{\label{fig:vmin_selected_diagnostics} Diagnostic plots for the selected predictive models of P spill.}\label{fig:selected_model_diagnostics_pspsill}
\end{figure}

\newpage



\codedataavailability{Analyses and figures in this manuscript were
drafted in RMarkdown. The RMarkdown scripts are available on the
corresponding author's GitHub page. All data used in this manuscript are
publicly available on local, state or federal data
portals.} %% use this section when having data sets and software code available



%%%%%%%%%%%%%%%%%%%%%%%%%%%%%%%%%%%%%%%%%%
%% optional

%%%%%%%%%%%%%%%%%%%%%%%%%%%%%%%%%%%%%%%%%%

%%%%%%%%%%%%%%%%%%%%%%%%%%%%%%%%%%%%%%%%%%

%%%%%%%%%%%%%%%%%%%%%%%%%%%%%%%%%%%%%%%%%%
\competinginterests{The authors declare no competing
interests.} %% this section is mandatory even if you declare that no competing interests are present

%%%%%%%%%%%%%%%%%%%%%%%%%%%%%%%%%%%%%%%%%%

%%%%%%%%%%%%%%%%%%%%%%%%%%%%%%%%%%%%%%%%%%
\begin{acknowledgements}
This manuscript emerged from dissertation work funded by Siskiyou County
SGMA planning grants, with funding from California water bonds.
\end{acknowledgements}

%% REFERENCES
%% DN: pre-configured to BibTeX for rticles

%% The reference list is compiled as follows:
%%
%% \begin{thebibliography}{}
%%
%% \bibitem[AUTHOR(YEAR)]{LABEL1}
%% REFERENCE 1
%%
%% \bibitem[AUTHOR(YEAR)]{LABEL2}
%% REFERENCE 2
%%
%% \end{thebibliography}

%% Since the Copernicus LaTeX package includes the BibTeX style file copernicus.bst,
%% authors experienced with BibTeX only have to include the following two lines:
%%
\bibliographystyle{copernicus}
\bibliography{library.bib}
%%
%% URLs and DOIs can be entered in your BibTeX file as:
%%
%% URL = {http://www.xyz.org/~jones/idx_g.htm}
%% DOI = {10.5194/xyz}


%% LITERATURE CITATIONS
%%
%% command                        & example result
%% \citet{jones90}|               & Jones et al. (1990)
%% \citep{jones90}|               & (Jones et al., 1990)
%% \citep{jones90,jones93}|       & (Jones et al., 1990, 1993)
%% \citep[p.~32]{jones90}|        & (Jones et al., 1990, p.~32)
%% \citep[e.g.,][]{jones90}|      & (e.g., Jones et al., 1990)
%% \citep[e.g.,][p.~32]{jones90}| & (e.g., Jones et al., 1990, p.~32)
%% \citeauthor{jones90}|          & Jones et al.
%% \citeyear{jones90}|            & 1990


\end{document}
