%% Copernicus Publications Manuscript Preparation Template for LaTeX Submissions
%% ---------------------------------
%% This template should be used for copernicus.cls
%% The class file and some style files are bundled in the Copernicus Latex Package, which can be downloaded from the different journal webpages.
%% For further assistance please contact Copernicus Publications at: production@copernicus.org
%% https://publications.copernicus.org/for_authors/manuscript_preparation.html

%% copernicus_rticles_template (flag for rticles template detection - do not remove!)

%% Please use the following documentclass and journal abbreviations for discussion papers and final revised papers.

%% 2-column papers and discussion papers
\documentclass[hess, manuscript]{copernicus}



%% Journal abbreviations (please use the same for preprints and final revised papers)

% Advances in Geosciences (adgeo)
% Advances in Radio Science (ars)
% Advances in Science and Research (asr)
% Advances in Statistical Climatology, Meteorology and Oceanography (ascmo)
% Annales Geophysicae (angeo)
% Archives Animal Breeding (aab)
% Atmospheric Chemistry and Physics (acp)
% Atmospheric Measurement Techniques (amt)
% Biogeosciences (bg)
% Climate of the Past (cp)
% DEUQUA Special Publications (deuquasp)
% Drinking Water Engineering and Science (dwes)
% Earth Surface Dynamics (esurf)
% Earth System Dynamics (esd)
% Earth System Science Data (essd)
% E&G Quaternary Science Journal (egqsj)
% EGUsphere (egusphere) | This is only for EGUsphere preprints submitted without relation to an EGU journal.
% European Journal of Mineralogy (ejm)
% Fossil Record (fr)
% Geochronology (gchron)
% Geographica Helvetica (gh)
% Geoscience Communication (gc)
% Geoscientific Instrumentation, Methods and Data Systems (gi)
% Geoscientific Model Development (gmd)
% History of Geo- and Space Sciences (hgss)
% Hydrology and Earth System Sciences (hess)
% Journal of Bone and Joint Infection (jbji)
% Journal of Micropalaeontology (jm)
% Journal of Sensors and Sensor Systems (jsss)
% Magnetic Resonance (mr)
% Mechanical Sciences (ms)
% Natural Hazards and Earth System Sciences (nhess)
% Nonlinear Processes in Geophysics (npg)
% Ocean Science (os)
% Polarforschung - Journal of the German Society for Polar Research (polf)
% Primate Biology (pb)
% Proceedings of the International Association of Hydrological Sciences (piahs)
% Safety of Nuclear Waste Disposal (sand)
% Scientific Drilling (sd)
% SOIL (soil)
% Solid Earth (se)
% The Cryosphere (tc)
% Weather and Climate Dynamics (wcd)
% Web Ecology (we)
% Wind Energy Science (wes)

% Pandoc citation processing

% The "Technical instructions for LaTex" by Copernicus require _not_ to insert any additional packages.
% 
% tightlist command for lists without linebreak
\providecommand{\tightlist}{%
  \setlength{\itemsep}{0pt}\setlength{\parskip}{0pt}}


%
\begin{document}


\title{Seasonal prediction of end-of-dry season watershed behavior in a
highly interconnected alluvial watershed, northern California}


\Author[1]{Claire}{Kouba}
\Author[1]{Thomas}{Harter}


\affil[1]{Department of Land, Air and Water Resources, University of
California, Davis, One Shields Avenue, Davis, CA, United States}

\runningtitle{Seasonal prediction of end-of-dry season watershed
behavior in a highly interconnected alluvial watershed, northern
California}

\runningauthor{Kouba and Harter}


\correspondence{Claire\ Kouba\ (cmkouba@ucdavis.edu)}



\received{}
\pubdiscuss{} %% only important for two-stage journals
\revised{}
\accepted{}
\published{}

%% These dates will be inserted by Copernicus Publications during the typesetting process.


\firstpage{1}

\maketitle


\begin{abstract}
In undammed watersheds in Mediterranean climates, the timing and
abruptness of the transition from the dry season to the wet season have
major implications for aquatic ecosystems. Of particular concern for
resource managers in many coastal areas is whether this transition can
provide sufficient flows at the right time to allow passage for spawning
anadromous fish, which is determined by dry season baseflow rates and
the timing of the onset of the rainy season. In (semi-) ephemeral
watershed systems, the dry season baseflow and rainy season onset timing
also dictate the timing of full reconnection of the stream system. In
this study, we propose methods to predict, approximately five months in
advance, two key hydrometeorologic metrics in the undammed rural Scott
River watershed in northern California. Both metrics are intended to
quantify the transition from the dry to the wet season, to characterize
the severity of a dry year and support seasonal adaptive management. The
first metric is the minimum 30-day dry season baseflow volume, which
occurs at the end of the dry season (September-October). The second
metric is the cumulative precipitation, starting Sept.~1st, necessary to
bring the watershed to a ``full'' or ``spilling'' condition
(i.e.~initiate the onset of wet season storm- or baseflows) after the
end of the dry season. As potential predictors of these two values, we
assess maximum snowpack, cumulative precipitation, the timing of the
snowpack and precipitation, spring groundwater levels, spring river
flows, reference ET, and a subset of these metrics from the previous
water year. We find that, though many of these predictors are correlated
with the two metrics of interest, of the predictors considered here, the
best prediction for both metrics is a linear combination of the maximum
snowpack water content and total October-April precipitation.
\textbf{These two linear models could reproduce historic values of
\(V_{min,~30~days}\) and \(P_{spill}\) with an average model error
(RMSE) of 1.4 Mm\textsuperscript{3} / 30 days (19.4 cfs) and 20.7 mm
(0.8 inches), respectively.-REVISE after new analysis } Although these
predictive indices could be used by governance entities to support local
water management, careful consideration of baseline conditions used as a
basis for prediction is necessary.
\end{abstract}


\copyrightstatement{The authors retain copyright for this publication.}


\section{Introduction}

In regions that experience periodic drought, such as the western United
States, indices summarizing hydroclimate or surface water supply
conditions are often critical decision-support tools for water managers
\citep[e.g.,][]{Garen1993}. An index can be forward-looking, such as
those that forecast near-term seasonal water supplies
\citep[e.g.,][]{Null2013, Verley2020}, or backward-looking, such as ones
that evaluate drought severity
\citep[e.g.,][]{Palmer1965, Guttman1998a, McKee1993, Wilhite1985, Wilhite2000}.
In California, forward-looking seasonal indices are used extensively by
water agencies to inform adaptive management decisions. The principal
examples are the Sacramento Valley Index (SVI) and San Joaquin Index
(SJI), which are seasonal forecasts used to determine water allocations
through the State Water Project
\citetext{\citealp{Null2013}; \citeyear[DWR][]{DWR2022}}. The state has
more recently published a retroactive categorical water year type (WYT)
dataset for sub-county level regions throughout California, based on a
weighted combination of the cumulative precipitation of the two
preceding water years (effectively, a partial one-year-holdover
provision), and assigning categorical types using percentiles within a
30-year ranking window \citeyearpar[DWR][]{DWR2021a}.

Complementing such summary indices, the functional flows approach is a
framework for providing a more detailed picture of the hydrologic
effects of water year type, climate change, human water use, and other
factors \citep[e.g.,][]{Poff1997, Bunn2002, Poff2010, Wheeler2018}. The
flows are ``functional'' because they serve an ecological purpose, such
as wet season flood flows, needed to disperse cottonwood seeds
\citep{Mahoney1998} and fall pulse flows, needed to provide passage for
spawning fall-run anadromous fish \citep{Moyle2002a}. A
California-specific functional flows framework has been developed to
assess the degree of hydrologic alteration between current and
unimpaired conditions \citep{Yarnell2020, Patterson2020}.

In this study, to test the utility of locally-tailored predictive
methods for hydrologic indices that incorporate functional flows, we
focus on a single basin, the Scott River watershed in northern
California (HUC8 18010208). We review the hydrologic indices and methods
currently used in decision-making, such as agricultural cropping choices
or regulatory water use restrictions, and propose two additional
decision-support metrics, both designed as quantitative forecasts. The
first is \(V_{min,~30~days}\), the minimum 30-day dry season baseflow
volume in a given water year, which typically occurs in September or
October. The second is a prediction of the cumulative rainffall needed
to wet up the watershed after the dry season such that subsequent
rainfall results in clear runoff events. This cumulative precipitation
depth is referred to as \(P_{spill}\). Both of these metrics have
significance for environmental flows and could support near-term
(seasonal) adaptive management, similar to the SVI and SJI in
California's Central Valley. Specifically, the magnitude of the minimum
baseflow rate sets the spatial extent of the aquatic ecosystem during
the dry season and influences rearing conditions for oversummering
juvenile salmonids \citep{Gorman2016}, while \(P_{spill}\) is related to
the timing of flows necessary for fall-run salmon passage: a greater
amount of rain needed to generate stormflow is correlated with
\textbf{to do: insert citation for this claim} a prolonged dry season,
which has delayed salmon access to spawning habitat in recent years
\citeyearpar[CDFW][]{CDFW2015a}. After defining and developing seasonal
predictions for \(V_{min,~30~days}\) and \(P_{spill}\), we then evaluate
trends over time and consider the effects that climate change and
changing water use patterns may have on the metrics considered in this
study, and the decisions they support.

\section{Methods}

In this study we used linear regression modeling to predict watershed
behavior at the end of the dry season (the response) using data
available the previous spring (the predictors). The Scott River
watershed has a snow-influenced Mediterranean climate, giving the
river's annual hydrograph a characteristic high-flow season during the
rainy winters, a gradual flow recession in the spring-summer as the
snowpack melts, and a low-flow dry season after the snowpack is depleted
(e.g., \autoref{fig:watershed_4states_illustration}). Water supplies for
agricultural and domestic use are relatively reliable in the Scott River
system {[}although some reports of dry wells occur in dry years;
Siskiyou County -\citet{SiskiyouCounty2021}{]}, and a key management
challenge is persistent low environmental flows during the dry season
baseflow period. In dry years, the lowest annual flowrates can overlap
with the spawning periods for fall-run anadromous fish, potentially
restricting fish passage and imperiling the long-term viability of the
Scott River fishery \citeyearpar[Siskiyou County][]{SiskiyouCounty2021}.
Post-1970s minimum dry season baseflows have been lower than pre-1977,
and very low minimums (\textless{} 10 cfs or 0.7 Mm\textsuperscript{3} /
30 days) have been more frequent in the past two decades
(\autoref{fig:fall_flows_data_exp}), making the management of these
flows more urgent.

This study focuses on the transition between the dry season and the wet
season, which at times can straddle the conventional water year boundary
of October 1st, and cumulative precipitation is used both as a predictor
and as a response variable (\(P_{spill}\)). When it is a predictor, a
traditional October 1st start date is used and it is summed as the
cumulative precipitation of October-April, to facilitate an end-of-April
prediction of fall conditions. When it is the response variable, to
capture uncommon September precipitation, cumulative precipitation is
counted starting on September 1st of the preceding water year. This
September 1st start date is also used in some graphs of climate and flow
data (e.g.~\autoref{fig:time_v_fall_rains_v_flow_fig}), to establish and
visualize baseline dry season conditions. Additionally, all flows in
this study are observed or simulated at the USGS Fort Jones streamflow
gauge (ID 11519500), a key monitoring location downstream of nearly all
water use and cultivated land in the HUC8 watershed
(\autoref{fig:watershed_fig_ch3}), with an observation record covering
water years 1942-2021.

\subsection{\texorpdfstring{Scott River watershed precipitation-runoff
behavior and
\(Q_{spill}\)}{Scott River watershed precipitation-runoff behavior and Q\_\{spill\}}}

To establish the context and meaning of the two proposed predictive
indices \(V_{min,~30~days}\) and \(P_{spill}\), a brief description of
the behavior of the watershed is necessary. In an undammed catchment,
the runoff response to one (or a series of) precipitation event(s) is
dependent on multiple factors, including antecedent soil moisture
conditions, the intensity and magnitude of the precipitation, and the
volume of water in aquifer storage \citep{Tarboton2003}. At a hillslope
scale, in areas where soil directly overlays (relatively) impermeable
bedrock and aquifer storage is not appreciably present, a threshold
runoff response to individual storm events has been observed: after a
certain quantity of rainfall, subsurface flow increases dramatically
\citep{Tromp-VanMeerveld2006}. The proposed mechanism is the filling and
connecting of various distributed storage volumes, such as soil pores
and microtopographic relief in the bedrock surface
\citep{Tromp-VanMeerveld2006}. Recently this concept has been extended
to the watershed or basin scale: relative to the beginning of a storm
event, a much higher flow response is possible only when a critical
number of storage volumes throughout a basin fill to a threshold level
and become connected \citep{McDonnell2021}.

In this study we expand this concept of a basin-scale, threshold-based
runoff response to the temporal scale of a season, rather than a single
storm event. In this framework the condition of the Scott River
watershed, as observed at a regional scale using the Fort Jones stream
gauge, can be classified in four main categories. These categories are
distinguished by current precipitation conditions and the volumetric
proportion of the hydrologically connected reservoirs that are full of
water (\autoref{tab:watershed_modes_tab}).

The term ``hydrologically connected reservoirs'' is defined here
functionally as storage that is sufficiently connected to the stream
system as to influence stream reactions to storm inflows. In this study
area only soil water and groundwater meet this definition. More
generally, water in the Scott River watershed is stored in five primary
reservoirs \citep{Harter2008}:

\begin{itemize}
\tightlist
\item
  snowpack
\item
  fractures in impermeable bedrock
\item
  soil moisture/subflow
\item
  the alluvial groundwater aquifer
\item
  streams and surface water bodies
\end{itemize}

Accumulating snowpack is present only in the mountainous areas of the
upper watershed, while the alluvial aquifer is present only within the
bounds of the groundwater basin underlying the flat valley floor; water
stored in fractured rock emerges as springs in the upper watershed
\citep{Mack1958} (\autoref{fig:watershed_fig_ch3}). In conditions with
sufficiently high soil water content or groundwater elevations, soil
moisture/subflow and groundwater become hydrologically connected to the
surface water system. Conversely, water in the snowpack and fractured
rock reservoirs is not hydrologically connected to major surface water
bodies until it melts or descends lower into the watershed, effectively
passing through the aquifer or soil to reach the stream. For convenience
the soil moisture/subflow and aquifer will be referred to as
``connected'' storage.

\begin{table*}[t]
\caption{Schematic of watershed behavior and functional flow components occurring during the transition from the dry season to the wet season in a Mediterranean climate; the categories are illustrated in an example annual hydrograph in Figure 1. Water storage level refers to the relative water content of the soil and aquifer within the watershed.}
\label{tab:watershed_modes_tab}
\begin{tabular}{p{1.8cm} p{1.6cm} p{4.8cm} p{5cm}}
\tophline
Water storage level & New precip. occurring? & Flow behavior description & Relevant functional flows \\
\middlehline
 Low & No & (A) Watershed draining from a medium-to-low storage level & Late spring recession and dry season baseflow \\ 
 \middlehline
Low & Yes & (B) Watershed filling from a low storage level, with muted response to new precipitation & Fall pulse flow or small/slow post-dry-season flow increase \\
\middlehline
High & No & (C) Watershed draining from a high storage level & Winter baseflow and early spring recession \\
\middlehline
High & Yes & (D) Watershed spilling from a high storage level, with rapid response to new precipitation & Winter stormflow\\
\bottomhline
\end{tabular}
\belowtable{}
\end{table*}

\begin{figure}
\includegraphics[width=1\linewidth]{f01} \caption{\label{fig:watershed_4states_illustration} Illustration of four categories of Scott River watershed behavior. The hydrograph in the highlighted periods demonstrates the following watershed behavior: A, dry season baseflow -- watershed draining from a low-to-medium storage level; B, moderate flow increase  -- muted hydraulic response to new precipitation; C, winter baseflow and early spring recession -- watershed draining from a high storage level; and D, winter stormflow -- rapid hydraulic response to new precipitation (storm spikes).}\label{fig:watershed_4states_illustration}
\end{figure}

\subsubsection{Rainfall-runoff response and functional flows}

At the end of the dry season, the watershed is in a ``draining from low
storage'' condition, which is reflected in a slowly declining or flat
hydrograph, with a flowrate that has decreased for several months
(\autoref{fig:watershed_4states_illustration}, first period A). As the
dry season ends, the watershed begins receiving rain, and enters a
condition of ``filling from a low storage level''. In this catchment,
much of the earliest water entering the system is routed as recharge
through the soil or the streambed to occupy space in the aquifer.
Because groundwater moves more slowly through the watershed than surface
water, the hydrograph at the Fort Jones gauge demonstrates a muted or
delayed response to early rain events
(\autoref{fig:watershed_4states_illustration}, period B).

At the onset of a new wet season, under average conditions, the flowrate
of filling is greater than the flowrate of draining, and so the
``filling from a low storage level'' condition at the beginning of a
rainy season is transient, lasting only until the filling process
occupies enough aquifer and soil storage volume to produce a ``full''
condition. After the water storage in the basin reaches ``full'', if no
more rain occurs, the watershed returns to its default ``draining''
condition, though from a higher storage baseline than during the dry
season, and with a higher draining flowrate
(\autoref{fig:watershed_4states_illustration}, first period C). If there
is additional precipitation, the hydraulic response is much more rapid,
reflecting a ``spilling'' condition
(\autoref{fig:watershed_4states_illustration}, intermittent events D).

The precipitation and winter temperatures during the wet season produce
an accumulation of snowpack, though in some years this can be reduced by
warm periods and rain-on-snow events. Melting snowpack contributes
subsurface flow and tributary streamflow to the lower watershed,
producing a spring flow recession typically lasting from the last major
precipitation event into the summer
(\autoref{fig:watershed_4states_illustration}, second period C and
second period A).

Many of the phenomena described in the above paragraphs have been
characterized as various types of functional flow
(\autoref{tab:watershed_modes_tab}). Winter stormflow is the obvious
functional flow metric corresponding to a ``spilling'' watershed. The
spring recession can last for three to six months and its steepness is
moderated by snowmelt. Because it bridges the high-storage and
low-storage states, the early and late spring recession appear in two
different flow behavior categories (\autoref{tab:watershed_modes_tab}).
Conversely, the flows classified under ``watershed filling from a low
storage level'' are somewhat ambiguous and dependent on year-to-year
conditions, since a discrete fall pulse flow does not occur in every
water year, and no distinct metric has been proposed for the small or
slow post-dry-season flow increases that constitute the watershed's
response to minor precipitation at the end of the dry season.

Given the regular behavior observed during the dry season-to-wet season
transition in the Fort Jones hydrograph, and the physical structure of
this highly inter-connected basin, we expect to find a flowrate
threshold at the Fort Jones gauge approximately defining the lower limit
of the ``full'' or ``spilling'' basin condition. This flowrate,
\(Q_{spill}\), was estimated to be 100 cfs based on visual inspection of
annual September-March hydrographs
(\autoref{fig:time_v_fall_rains_v_flow_fig}, panel A). \textbf{Insert
other analysis of QSpill}

\subsubsection{Stream-aquifer interaction}

In the groundwater basin portion of the watershed, the alluvial aquifer
is the largest storage reservoir. Groundwater-surface water interactions
drive Scott River flow behavior towards the end of the dry season,
before the next rainy season begins, when snowpack is depleted and
streamflow in many areas is sustained by groundwater discharge alone.
Discharge to streams from the alluvial aquifer occurs along the thalweg
of the Scott River. In this highly-interconnected system, groundwater
discharge in one reach of the river is typically approximately balanced
out by infiltration through the streambed to the aquifer, much of it
occurring on the upper alluvial fans of the tributary streams (see
discussion below).

We can use the Scott Valley Integrated Hydrologic Model {[}SVIHM;
\citet{Tolley2019}{]} to obtain the estimated volume of water exchanged
monthly, in water years 1991-2018, between the surface stream network
and the underlying aquifer. All positive fluxes and all negative fluxes
(corresponding to gaining and losing stream reaches) were summed
independently and then added to create a net value for each month in the
simulation period (\autoref{fig:flow_to_aq_and_stream}, upper panel).
These net monthly groundwater-to-stream flux values were then compared
to simulated monthly flow volumes in the Scott River, measured at the
Fort Jones gauge (\autoref{fig:flow_to_aq_and_stream}, lower panel).

\subsection{\texorpdfstring{Observed response variables
(\(V_{min,~30~days}\) and
\(P_{spill}\))}{Observed response variables (V\_\{min,\textasciitilde30\textasciitilde days\} and P\_\{spill\})}}

The Scott River is an undammed watershed, in which estimates of annual
precipitation are an order of magnitude greater than the estimated
combined volume of water stored in surface water bodies or aquifers and
water pumped or diverted for agriculture \citep{Tolley2019}. In this
study we tested whether fundamental hydrologic characteristics,
specifically dry-season draining behavior and hydraulic response to
early wet season cumulative precipitation, can be predicted using
observable hydroclimate data. The first step is quantification of the
two response variables.

\subsubsection{\texorpdfstring{Dry season baseflow quantities
(\(V_{min,~30~days}\)) and
timing}{Dry season baseflow quantities (V\_\{min,\textasciitilde30\textasciitilde days\}) and timing}}

Multiple numerical summaries of dry season baseflows were evaluated for
suitability as the response variables in this prediction exercise (e.g.,
monthly flow volumes in \autoref{fig:fall_flows_data_exp}). Monthly flow
volumes were preferred over a minimum daily flow value to represent
durable conditions at the end of a dry season, and to reduce the
influence of individual events that might affect flow on one or a small
number of days, such as groundwater pumping or surface water diversions.

Historically, the rainy season in California tends to begin in October,
and so by convention each water year begins on October 1st of the
previous calendar year, and ends on September 30th. Matching this
convention, in most years, the minimum-flow month for the Scott River is
September; however, uncommon September storms can elevate flow volumes,
and in some years with a late rainy season onset, the October flow
volume may be lower. To capture these dynamics, for each calendar year,
we calculated a rolling 30-day sum of daily flow volumes in the period
July-December to identify the 30-day period with the minimum flow,
referred to as \(V_{min,~30~days}\) (\autoref{fig:fall_flows_data_exp}).
For consistency, each annual \(V_{min,~30~days}\) value was assigned to
the water year ending in September of that calendar year, even if the
minimum flow window included days in October of the following water
year.

\subsubsection{\texorpdfstring{Cumulative precipitation
\(P_{spill}\)}{Cumulative precipitation P\_\{spill\}}}

\(P_{spill}\) was calculated for each water year as the cumulative
rainfall at the end of a dry season, starting September 1st, recorded on
the first day that the Fort Jones gauge measured flow greater than
\(Q_{spill}\) (\autoref{fig:fall_flows_data_exp}, lower panel). As
stated above, conceptually, it is the amount of rainfall needed to
``fill'' the watershed such that it responds rapidly to new
precipitation.

A dry season can have negative effects on an aquatic ecosystem if it
produces extraordinarily low flows or if it lasts for an extraordinarily
long time \citeyearpar[e.g., delayed salmon habitat access documented in
CDFW][]{CDFW2015a}. The quantity \(P_{spill}\) is correlated with both a
lower minimum flow volume and a later river reconnection
(\autoref{fig:p_spill_vs_baseflow_and_recon_timing}). If predicted in
advance, a forecasted value of \(P_{spill}\) would be an indicator of
the risk of a severe dry season. The timing of the increase in dry
season baseflows has trended later over the past several decades
\citeyearpar[see Siskiyou County][]{SiskiyouCounty2021}, and there could
be demand for seasonal predictions of onset of the coming rainy season.
However, predicting the timing of the onset of the rainy season or of
\(Q_{spill}\) would likely rely on uncertain long-term weather forecasts
and is beyond the scope of this paper. In other words, due to randomness
in rainfall timing, the exact dry season baseflow extension caused by a
higher \(P_{spill}\) is highly variable and, hence, unpredictable.

\begin{figure}
\includegraphics[width=1\linewidth]{f02} \caption{\label{fig:p_spill_vs_baseflow_and_recon_timing} The quantity P spill (i.e., the amount of rainfall needed to 'fill' the watershed such that it 'spills', or responds rapidly to new precipitation) is correlated with both a lower minimum dry season baseflow volume (top panel) and a later date of river reconnection (lower panel).}\label{fig:p_spill_vs_baseflow_and_recon_timing}
\end{figure}

\subsection{Potential predictors and selected formulations}

To evaluate candidate predictors of dry season baseflows, Pearson's
correlation coefficient, \(R\), was calculated between observed response
variables \(V_{min,~30~days}\) and \(P_{spill}\), and the following
categories of observed predictor data (\autoref{fig:corr_matrix}):

\begin{enumerate}
\def\labelenumi{\arabic{enumi}.}
\tightlist
\item
  Spring (March-May) water level observations in each of 74 individual
  wells (\autoref{fig:gw_vs_fall_flows_corr_map}).
\item
  Annual maximum snowpack water content at each individual snow
  monitoring station at 20 CDEC stations
  (\autoref{fig:watershed_fig_ch3}).
\item
  Cumulative precipitation, October-April, at each weather station
  within the watershed, and five outside the watershed (total of 17 NOAA
  stations). In these records, missing values (i.e., days with no
  recorded observation) are assumed to have 0 precipitation. Water years
  with more than 5 missing days are excluded from the predictor dataset
  (\autoref{fig:watershed_fig_ch3}).
\item
  Cumulative precipitation, October-April, of a composite precipitation
  record with no missing values, representing the mean of the Callahan
  and Fort Jones NOAA weather stations (located at the southern and
  northern ends of the valley, respectively), and referred to as
  ``cal\_fj\_interp''. To generate the composite record, missing values
  in the Callahan and Fort Jones station records were interpolated based
  on observations at neighboring stations \citep[see method
  in][]{Foglia2013a}.
\item
  The flow volumes observed at the Fort Jones gauge (USGS ID 11519500)
  during the preceding March and April
  (\autoref{fig:watershed_fig_ch3}).
\item
  Cumulative reference evapotranspiration (ET\textsubscript{0}),
  October-April, using observations from the Scott Valley CIMIS station,
  Station No.~225 (2015-2021), or Spatial CIMIS estimates of
  ET\textsubscript{0} at the location of Station 225 (2002-2015)
  (\autoref{fig:watershed_fig_ch3}).
\item
  The timing (in Julian days) of the date of maximum snowpack
  measurement.
\item
  The timing (in Julian days) of the date of the volumetric center of
  the rainy season, calculated as the day the cumulative precipitation
  crossed 50\% of the total.
\item
  The 1-year-lagged metrics of maximum snowpack, October-April
  cumulative precipitation, and April water levels (e.g., the
  October-April cumulative precipitation measured a full 17-23 months
  prior to a September minimum flow).
\end{enumerate}

Individual measuring locations, such as wells or weather stations, were
evaluated for sample size (i.e., years of data) and degree of
relatedness with the two response variables. Relatedness of the
monitoring locations with the highest \(R\) values in each category of
monitoring observation are shown in \autoref{fig:corr_matrix}.

\begin{figure}
\includegraphics[width=1\linewidth]{f03} \caption{\label{fig:watershed_fig_ch3} Scott River HUC8 watershed and groundwater basin boundaries, stream network, and key monitoring locations: the Fort Jones stream gauge (USGS ID 11519500), weather stations, snow observation locations, and CIMIS station. Selected locations are highlighted with an enlarged symbol and an abbreviated label.}\label{fig:watershed_fig_ch3}
\end{figure}

\subsubsection{\texorpdfstring{Prediction formulae for
\(V_{min,~30~days}\) and
\(P_{spill}\)}{Prediction formulae for V\_\{min,\textasciitilde30\textasciitilde days\} and P\_\{spill\}}}

With a sample size of 80 years of dry season baseflow volumes, a one- or
two-predictor model is best to avoid overfitting \citep{James2013}. To
predict \(V_{min,~30~days}\), a set of six one-predictor models were
generated using the observation location from each category with the
highest \(R\), and model fit was evaluated using Leave One Out Cross
Validation {[}LOOCV; \citet{James2013}{]}
(\autoref{fig:one_predictor_model}). For a dataset with \(n\)
observations, the LOOCV error of a predictive model is obtained by
recalculating the model coefficients \(n\) times, each time leaving out
one observation, and comparing the resulting prediction to the single
left-out observation. The root mean square of these \(n\) errors is the
LOOCV error used to evaluate model performance in Results.

The single predictors with the lowest LOOCV error (excluding Reference
ET, due to insufficient observation record length) were used to produce
a set of four two-predictor models (\autoref{fig:two_predictor_model})
for \(V_{min,~30~days}\), including two that incorporate a partial
one-year holdover term, to test the validity of the DWR Water Year Type
index method in this local setting. A similar approach was used to
assess two-predictor models for \(P_{spill}\), though no one-year
holdovers were included, and several additional two-predictor
combinations were evaluated. In both cases, the best-performing model
took the following form:

\[Predicted_{i} = Int. + m_A * obs_{A,~i}+m_B*obs_{B,~i}\]

Where:

\begin{itemize}
\tightlist
\item
  \(Predicted_i\) is the predicted value (either \(V_{min,~30~days}\) or
  \(P_{spill}\)) in calendar year \(i\) (i.e., at the end of water year
  \(i\)).
\item
  \(obs_{A,~i},~obs_{B,~i}\) are the observed predictor values in
  October-April in water year \(i\) (millimeters).
\item
  \(Int.,~m_A,~m_B\) are the coefficients of the selected linear model.
\end{itemize}

\section{Results}

\subsection{Scott River precipitation-runoff behavior}

Visual inspection of 80 years of Fort Jones hydrograph behavior during
the transition from the dry season to the rainy season indicates that
there are two distinct domains of flow: one in which flow is relatively
flat (dry season baseflow), and one in which the flowrate is an order of
magnitude higher, and it is highly responsive to rain events (wet season
baseflow and stormflow; \autoref{fig:time_v_fall_rains_v_flow_fig},
panel A). By visual inspection, and corroborating local observations
(see discussion below), the approximate boundary between these domains,
denoted as \(Q_{spill}\), is 100 cfs (approximately 7.5
Mm\textsuperscript{3} per month). The intermediate hydrologic state,
``filling from low storage'', is visible in some fall-winter hydrographs
(\autoref{fig:time_v_fall_rains_v_flow_fig}, panel A), but tends to last
a relatively short time before the filling rate overwhelms the draining
rate and produces a responsive ``full'' condition.

\begin{figure}
\includegraphics[width=1\linewidth]{f04} \caption{\label{fig:time_v_fall_rains_v_flow_fig} In all three panels, 80 years of data series from September 1 to March 31 are overplotted to illustrate dynamics during the transition from the dry to the wet season: observed Fort Jones hydrographs in Panel A; cumulative rainfall and Fort Jones flow values on fall and winter days in Panel B; and cumulative rainfall over time in Panel C.}\label{fig:time_v_fall_rains_v_flow_fig}
\end{figure}

Monthly volume of stream-aquifer exchange, estimated using SVIHM
\citep{Tolley2019}, can be used to further investigate baseflow
generation and the boundaries between the draining and spilling flow
domains. In most months, the aquifer discharge and stream leakage
components of the exchange tend to be of equivalent volume, and net
stream-aquifer exchange near 0 (\autoref{fig:flow_to_aq_and_stream},
upper panel). Exceptions to this tend to happen only at high Scott River
flowrates; all net groundwater-to-stream flux volumes of
\textgreater0.25 Mm\textsuperscript{3} / 30 days (approximately 3.3 cfs)
occur at simulated Fort Jones flowrates of \textgreater20
Mm\textsuperscript{3} (approximately 267 cfs;
\autoref{fig:flow_to_aq_and_stream}, lower panel).

Additionally, net monthly stream-aquifer exchange volume tends to be an
order of magnitude lower than the flow simulated at the Fort Jones
gauge. Clear seasonal trends in the net exchange volume suggest that in
the winter and spring, precipitation events can temporarily produce
large pulses in groundwater discharge. In the summer growing season,
when flows are high (e.g.~\textgreater{} 10 Mm\textsuperscript{3}/month,
during the early summer snowmelt period), the result tends to be net
aquifer recharge, but at low flowrates, the surface water flow is
sustained by groundwater discharge. Similarly, very low dry season flows
(e.g., \textless{} 1 Mm\textsuperscript{3}/month) are largely composed
of groundwater discharge, but when flowrates are higher the direction of
net stream-aquifer exchange is more variable, responding to the
elevation of the proximate groundwater
(\autoref{fig:flow_to_aq_and_stream}, lower panel).

\begin{figure}
\includegraphics[width=1\linewidth]{f05} \caption{\label{fig:flow_to_aq_and_stream} Both stream leakage and aquifer discharge increase in the rainy season, while net flux to the stream remains relatively close to 0 (top panel). Strong seasonal trends are evident in net flux to the stream (lower panel; described further in text).}\label{fig:flow_to_aq_and_stream}
\end{figure}

\subsection{Observed response variables}

\subsubsection{Dry season baseflow quantity and timing, and Scott River
eras}

Minimum 30-day dry season baseflow volumes, denoted here as
\(V_{min, ~30~days}\), have ranged from 0.3 to 7.5 Mm\textsuperscript{3}
/ 30 days, with one outlier value of 13.9 Mm\textsuperscript{3} / 30
days in 1984, when an early September storm followed a wet year in 1983
(\autoref{fig:fall_flows_data_exp}).

Three periods of water use and climate forces have been proposed for the
Scott River \citep[e.g., by][]{Pyschik2022}: Eras 1, 2, and 3, ranging
from 1942-1976, 1977-2000, and 2001-2021, respectively. These eras are
separated by the low minimum flow in the year 1977, which corresponds to
the widespread installation of groundwater pumps, and by the onset of a
two-decade abnormally dry period in 2000.

Matching other long-term declining flow trends in this watershed, the
flows in August and September are relatively steady in Era 1, and they
become more variable with significantly lower lows in Eras 2 and 3
(minimum of 2.1 Mm\textsuperscript{3} / 30 days {[}28.6 average cfs{]}
in 1942-1976 and 0.3 Mm\textsuperscript{3} / 30 days {[}4.4 average
cfs{]} in 1977-2021; \autoref{fig:fall_flows_data_exp}, upper panel).

The timing of the midpoint of the 30-day minimum-flow period falls most
commonly in September, though it has fluctuated over the last eight
decades (\autoref{fig:fall_flows_data_exp}, middle panel).

\begin{figure}
\includegraphics[width=1\linewidth]{f06} \caption{\label{fig:fall_flows_data_exp} FJ Gauge flow volume, by year, aggregated to monthly time windows in the late summer, fall, and early winter. Eras are noted that correspond to various management and climate forces (e.g., the widespread installation of groundwater wells in the late 1970s, and the onset of a two-decade abnormally dry period in 2000). }\label{fig:fall_flows_data_exp}
\end{figure}

\subsubsection{Cumulative fall precipitation and watershed response}

In some water years prior to the 1980s, the Fort Jones flowrate exceeded
\(Q_{spill}\) on September 1st
(\autoref{fig:time_v_fall_rains_v_flow_fig}, panels A and B), indicating
that even under persistent dry season draining conditions, under the
climate and water use conditions of wet years in the mid-20th century,
the Scott River remained responsive to new precipitation year-round. As
a result, the range of \(P_{spill}\), the cumulative precipitation
necessary to reach \(Q_{spill}\), is wide (0 to 178 mm, or 0 to 7
inches) (\autoref{fig:fall_flows_data_exp}, lower panel). Mean
\(P_{spill}\) values were 45, 70, and 68 mm in Eras 1, 2 and 3,
respectively.

\subsection{\texorpdfstring{Predictor comparison for
\(V_{min.~30~days}\) and
\(P_{spill}\)}{Predictor comparison for V\_\{min.\textasciitilde30\textasciitilde days\} and P\_\{spill\}}}

\begin{figure}
\includegraphics[width=1\linewidth]{f07} \caption{\label{fig:corr_matrix} Correlation coefficient matrix of two response variables, minimum 30-day dry season baseflow volumes (V min) and cumulative precipitation necessary to produce 100 cfs in the Scott River (P spill), with various possible predictor metrics. Gray, purple, and blue squares highlight the inter-category correlation coefficients of snowpack metrics, Oct-April cumulative precipitation, and March-May groundwater elevation measurements. Red and yellow rectangles highlight the predictors with the greatest absolute correlation coefficient values with V min and P spill, respectively.}\label{fig:corr_matrix}
\end{figure}

The observations of spring flows, snowpack, valley floor precipitation,
and groundwater elevation are positively correlated both within each
category and to each other overall, which is unsurprising: wet years are
associated with higher values in all of these categories. Groundwater
wells with highest predictive power tend to have long records (e.g.,
\(n\) of 10 or greater years) and to be close to the Scott River
(\autoref{fig:gw_vs_fall_flows_corr_map}); these results focus on two
wells proximate to the river, with long records (well IDs
415635N1228315W001 and 416295N1228926W001).

Both response variables are strongly correlated with four categories of
observations: spring flowrates, maximum snow water content, cumulative
precipitation recorded at weather stations or or near the valley floor
(October-April), and March-May water levels in some wells. Observations
in these categories are positively correlated with \(V_{min.~30~days}\)
and negatively correlated with \(P_{spill}\). The correlation
coefficient, \(R\), of these response-predictor relationships range from
0.5 to 0.73 for \(V_{min.~30~days}\) and from -0.45 to -0.76 for
\(P_{spill}\) (\autoref{fig:corr_matrix}).

Conversely, cumulative ET\textsubscript{0}, October-April is negatively
correlated with \(V_{min.~30~days}\) and positively correlated with
\(P_{spill}\) (\(R\) of -0.68 and 0.65 for \(V_{min.~30~days}\) and
\(P_{spill}\), respectively). October-April cumulative
ET\textsubscript{0} is also negatively correlated with snow,
precipitation, and groundwater level indicators. While ET can remove a
significant volume of water from the watershed, this correlation
reflects the fact that years with more rainy or stormy days accumulate
less total insolation and atmospheric water demand, rather than
indicating that high ET is driving low flows. Additionally, the
relatively high absolute values of \(R\) for between ET\textsubscript{0}
and the two response variables may be due to a small sample size, as all
available ET\textsubscript{0} observations or estimates were collected
in 2002 or later (i.e., in Era 3; \autoref{fig:one_predictor_model}).

Both response variables are also somewhat correlated with snow timing
(i.e., the Julian day of the maximum measured snowpack in a given year;
\(R\) of 0.33 to 0.52 and -0.24 to -0.42 for \(V_{min.~30~days}\) and
\(P_{spill}\), respectively), but no significant correlation is evident
between the response variables and precipitation timing
(\autoref{fig:corr_matrix}).

A subset of observations from the previous water year were included in
the correlation matrix to test for multi-year influence on the response
variables. These previous-year metrics had a slight positive correlation
with \(V_{min.~30~days}\) (\(R\) of 0.29 to 0.33), and an even slighter
negative correlation with \(P_{spill}\) (\(R\) of -0.11 to -0.24),
providing moderate evidence for an ``echo'' effect of the previous
year's hydroclimate conditions on a given fall season.

\begin{figure}
\includegraphics[width=1\linewidth]{f08} \caption{\label{fig:gw_vs_fall_flows_corr_map} Boundary of the groundwater basin (corresponding approximately to the extent of the flat valley floor in the Scott River watershed) and selected well locations. Colors correspond to the correlation coefficients between April groundwater elevations and September flow volume. The wells included in the predictor comparison are highlighted with a red outer square.}\label{fig:gw_vs_fall_flows_corr_map}
\end{figure}

\subsection{\texorpdfstring{Predicted values of
\(V_{min.~30~days}\)}{Predicted values of V\_\{min.\textasciitilde30\textasciitilde days\}}}

\subsubsection{Predictor assessment and prediction formula}

In each of six high-\(R\) categories, the monitoring location in each
category with the highest \(R\) value with observed \(V_{min.~30~days}\)
values was selected for further analysis
(\autoref{fig:one_predictor_model}). Out of this set of six, the maximum
snowpack and October-April cumulative precipitation produce the lowest
model errors (LOOCV errors of 2.74 and 2.72 Mm\textsuperscript{3},
respectively; \autoref{fig:one_predictor_model}, top two panels). In
combinations of two predictors, a linear combination of maximum snowpack
and cumulative precipitation improved on the best single-predictor
model, with an LOOCV error of 2.29 Mm\textsuperscript{3}
(\autoref{fig:two_predictor_model}, upper left panel).

Among the two-predictor models evaluated was a combination of maximum
snowpack water content and the timing of the maximum measurement
(\autoref{fig:two_predictor_model}, top right panel). This produced a
slightly larger error (2.78 Mm\textsuperscript{3}) than the
single-predictor model with maximum snowpack water content alone (2.74
Mm\textsuperscript{3}; \autoref{fig:one_predictor_model}, middle left
panel), indicating that the timing of maximum snow accumulation is
either relatively unimportant in generating dry season baseflows --
perhaps because, regardless of the peak time, the melting snowpack
becomes recharge, which moves slowly enough through the subsurface to
buffer the timing of snowmelt -- or that the actual timing of snowpack
maximum is not captured in temporally sparse snow course measurements.

Additionally, two models featuring a partial one-year holdover were
evaluated, to test the validity of this component of the methodology of
DWR's Water Year Type index. In both cases, the addition of the climate
data from the previous year produced a very small change in model error
(\autoref{fig:two_predictor_model}, two lower panels), indicating that
in the Scott Valley context, the previous year's climate may have a
minor influence on dry season flows relative to the immediately
preceding rainy season.

Based on these results, the model selected as the \(V_{min.~30~days}\)
prediction formulation was a linear combination of snowpack maximum from
the Swampy John (SWJ) snow station (with data collected by CDEC) and
cumulative October-April precipitation from the Fort Jones Ranger
Station (FJRS) weather station (with data collected by NOAA) as follows:

\[V_{min.,~30~days,~i} = -1.33 + 0.00525 * FJRS_{Oct-Apr,~i}+0.00267*SWJ_{max,~i}\]

Where:

\begin{itemize}
\tightlist
\item
  \(V_{min.,~30~days,~i}\) is the predicted value of minimum 30-day dry
  season baseflows in calendar year \(i\) (i.e., at the end of water
  year \(i\)) (million m\textsuperscript{3} or Mm\textsuperscript{3})
\item
  \(SWJ_{max,~i}\) is the maximum snow water content recorded at the
  Swampy John snow course (CDEC station ID SWJ or 285) in water year
  \(i\) (millimeters)
\item
  \(FJRS_{Oct-Apr,~i}\) is the cumulative precipitation, recorded
  October-April of water year \(i\), measured at the Fort Jones Ranger
  Station (NOAA station ID USC00043182) (millimeters)
\end{itemize}

\begin{figure}
\includegraphics[width=1\linewidth]{f09} \caption{\label{fig:one_predictor_model} Single-predictor models of minimum 30-day dry season baseflows in the Scott River.}\label{fig:one_predictor_model}
\end{figure}

\begin{figure}
\includegraphics[width=1\linewidth]{f10} \caption{\label{fig:two_predictor_model} Two-predictor models of minimum 30-day dry season baseflows in the Scott River.}\label{fig:two_predictor_model}
\end{figure}

\subsubsection{\texorpdfstring{Predicted and observed
\(V_{min.~30~days}\) over
time}{Predicted and observed V\_\{min.\textasciitilde30\textasciitilde days\} over time}}

\begin{figure}
\includegraphics[width=1\linewidth]{f11} \caption{\label{fig:v_min_over_time} Observed and predicted minimum 30-day dry season baseflows both trend downward between the three eras of the period of record (top panel). The predicted-minus-observed difference (residual) over time also reflects this trend, underpredicting minimum flows pre-1977 and overpredicting them post-2000 (middle panel). The predictive model is based on observations from the full record, but three additional models were generated based on only the observations from Eras 1, 2, and 3. Residuals based on Era 1 data are similar to those of the full record; Era 2 residuals tend to overpredict more than the full record; and Era 3 residuals show better performance post-2000 than the full record, but significant underprediction pre-2000.}\label{fig:v_min_over_time}
\end{figure}

The \(V_{min.~30~days}\) formulation proposed above predicts minimum
30-day dry season baseflows with a model error of 2.3
Mm\textsuperscript{3} (31.3 cfs), and a root mean squared error of 1.4
Mm\textsuperscript{3} (19.4 cfs).

Matching the historical trends of decreasing snowpack, the observed and
predicted \(V_{min.~30~days}\) values show a downward trend over time
(\autoref{fig:v_min_over_time}, top panel). An outlier in the year 1984
reflects extremely high minimum dry season baseflows, relative to the
predicted values and the overall distribution. In that year, a
relatively high-baseflow season was followed by an early September
storm. The model residual (predicted minus observed flow volumes) for
this year is also an outlier, indicating that the model has a sufficient
sample size to not be overwhelmed by this extreme value produced by an
extremely uncommon sequence of events (\autoref{fig:v_min_over_time},
middle panel).

The predictive \(V_{min.~30~days}\) model is based on observations from
the full record, but three additional models were generated based on
only the observations from each period: Eras 1, 2, and 3, respectively.
Residuals based on Era 1 data are similar to those of the full record,
with a slight but systematic overprediction in Era 3; Era 2 residuals
tend to overpredict in Era 1 more than the full record; and Era 3
residuals offer better performance in Era 3 than the full record, but
produce significant systematic underpredictions pre-2000
(\autoref{fig:v_min_over_time}, middle panel).

\subsection{\texorpdfstring{Predicted values of
\(P_{spill}\)}{Predicted values of P\_\{spill\}}}

\subsubsection{Predictor assessment and prediction formula}

The results of the predictor assessment for the \(P_{spill}\) prediction
formula were similar to those for \(V_{min.~30~days}\), in that the two
best single predictors were October-April cumulative precipitation and
maximum snowpack (\autoref{fig:one_predictor_model_p_spill}, top two
panels), with LOOCV model errors of 695 and 496 mm, respectively.
(Reference ET was once again excluded from consideration based on a
short record length.) Again similar to \(V_{min.~30~days}\), the best
two-predictor model was the combination of the two best single
predictors, with an LOOCV error of 461 mm
(\autoref{fig:two_predictor_model_p_spill}, upper left panel).

Several combinations of correlated observation categories produced
comparable model results, such as spring water levels with maximum snow,
maximum snow timing, and cumulative precipitation
(\autoref{fig:two_predictor_model_p_spill}, upper right and two middle
panels). However, not all combinations of co-correlated data produced
reasonable predictors; a model with a linear combination of maximum
snowpack timing and March flow volumes performed relatively poorly
(LOOCV error of 1,005 mm; \autoref{fig:two_predictor_model_p_spill},
lower right panels).

Based on these results, the model selected as the \(P_{spill}\)
formulation for a given water year was a linear combination of the same
observation records as \(V_{min.~30~days}\): snowpack maximum from the
SWJ snow station (with data collected by CDEC) and cumulative
October-April precipitation from the Fort Jones Ranger Station weather
station (station ID USC00043182, with data collected by NOAA).

\[P_{spill,~i} = 123 -0.111 * FJRS_{Oct-Apr,~i} - 0.0274* SWJ_{max,~i}\]

Where:

\begin{itemize}
\tightlist
\item
  \(P_{spill,~i}\) is the predicted value of cumulative rainfall at the
  end of the dry season, starting Sep.~1, on the first day that the Fort
  Jones gauge records flow greater than or equal to 100 cfs in calendar
  year \(i\) (i.e., at the end of water year \(i\)) (millimeters)
\item
  \(SWJ_{max,~i}\) is the maximum snow water content recorded at the
  Swampy John snow course (CDEC station ID SWJ or 285) in water year
  \(i\) (millimeters)
\item
  \(FJRS_{Oct-Apr,~i}\) is the cumulative precipitation, recorded
  October-April of water year \(i\), measured at the Fort Jones Ranger
  Station (NOAA station ID USC00043182) (millimeters)
\end{itemize}

\begin{figure}
\includegraphics[width=1\linewidth]{f12} \caption{\label{fig:one_predictor_model_p_spill} Single-predictor models of P spill, the cumulative precipitation after the dry season needed to generate 100 cfs of flow in the Scott River.}\label{fig:one_predictor_model_p_spill}
\end{figure}

\begin{figure}
\includegraphics[width=1\linewidth]{f13} \caption{\label{fig:two_predictor_model_p_spill} Two-predictor models of P spill, the cumulative precipitation after the dry season needed to generate 100 cfs of flow in the Scott River.}\label{fig:two_predictor_model_p_spill}
\end{figure}

\subsubsection{\texorpdfstring{Predicted and observed \(P_{spill}\) over
time}{Predicted and observed P\_\{spill\} over time}}

The \(P_{spill}\) estimate formulation proposed above predicts
\(P_{spill}\) values with a model LOOCV error of 461 mm (18.1 inches),
and a root mean squared error of 20.7 mm (0.8 inches).

Matching the historical trends of decreasing snowpack, the observed and
predicted \(P_{spill}\) values show an increasing trend over time
(\autoref{fig:pspill_pred_over_time}, top panel). A high outlier in
calendar year 1994 (in early water year 1995) was caused by a dry water
year 1994 followed by a series of small storms in November and December,
none of which produced 100 cfs of flow, followed by a much larger storm
on January 8th-9th of 1995 in which the river flow jumped to 600 and
then 7,500 cfs in two days.

The predictive \(P_{spill}\) model is based on observations from the
full record, but three additional models were generated based on only
the observations from each period: Eras 1, 2, and 3, respectively.
Residuals based on Era 1 tend to underpredict Eras 2 and 3 more than the
full-record model; Era 2 residuals tend to overpredict in Eras 1 and 3
more than the full record; and Era 3 residuals have a slightly higher
tendency to underpredict than the full record, but overall are fairly
similar to the full-record residuals (\autoref{fig:v_min_over_time},
lower panel).

\begin{figure}
\includegraphics[width=1\linewidth]{f14} \caption{\label{fig:pspill_pred_over_time} Observed and predicted values of P spill (top panel) indicate a worse model fit for the P spill prediction than for minimum 30-day dry season baseflows (Figure 9). Serious overprediction in Era 1 is followed by more mixed over- and under-prediction in Eras 2 and 3 (bottom panel). The overall P spill model is based on observations from the full record, but three additional models were generated based on only the observations from Eras 1, 2, and 3. Residuals based on Era 1 data are generally lower than those from Eras 2 or 3 or from the full record.}\label{fig:pspill_pred_over_time}
\end{figure}

\subsection{Comparison with California DWR Water Year Type (WYT)
category}

\begin{figure}
\includegraphics[width=1\linewidth]{f15} \caption{\label{fig:resp_vars_wyt} DWR Water Year Type indices over time and compared to the two metrics of hydrologic conditions developed in this study: minimum 30-day dry season baseflow volume (V min) and the amount of precipitation necessary to produce 100 cfs flow in the Scott River (P spill).}\label{fig:resp_vars_wyt}
\end{figure}

The DWR water year type categories map fairly well onto the two proposed
hydrologic indices \(V_{min.~30~days}\) and \(P_{spill}\)
(\autoref{fig:resp_vars_wyt}, upper two panels), which is to be
expected, as both DWR WYT and the two proposed indices are based in part
on cumulative precipitation data. However, there is less of an ability
to identify a long-term trend in the DWR WYT index time series than in
the time series of observed or predicted \(V_{min.~30~days}\) or
\(P_{spill}\) values. Likely causes include the information lost when
binning water years into five categories, and the 30-year ranking window
that would prevent a direct comparison of post-2000 WYTs with pre-1950s
WYTs (\autoref{fig:resp_vars_wyt}, lower panel).

\section{Discussion}

\subsection{Scott River watershed behavior}

The degree to which these forward-looking seasonal predictions are
accurate depends on fundamental hydrologic relationships between climate
inputs and flow outputs, with some complications introduced by water
evaporating or transpiring through native or cultivated vegetation. The
condition of a ``full'' watershed can be operationally defined as a
highly responsive to new precipitation. The condition is transient, and
proximity to a full condition relies on the balance of slow draining and
rapid filling flowrates in any given rainy season. However, in this
Mediterranean climate, the general shape of the relationship between
cumulative precipitation-runoff behavior is preserved in dry and wet
water years (\autoref{fig:time_v_fall_rains_v_flow_fig}, panel B).

Although a \(Q_{spill}\) of 100 cfs was identified by visual inspection
of aggregate fall hydrograph behavior
(\autoref{fig:time_v_fall_rains_v_flow_fig}), it also matches
information from local stakeholders. Many tributary streams on the
valley floor run dry during the summer and fall, and some tributary
streams respond more quickly to fall precipitation than others.
Generally, the timing of all tributaries reaching flowing status
corresponds with the Fort Jones gauge reaching 100 cfs
\citep{Sommarstrom2020}.

Simulated estimates of stream-aquifer exchange corroborates these
precipitation-flow relationships. Dry season baseflow
(\(V_{min.~30~days}\)) and the onset of wet season flow (framed in terms
of \(P_{spill}\)) are both influenced by net groundwater discharge from
the aquifer. One interpretation of the high frequency of near-0 net
monthly stream-aquifer flux values (\autoref{fig:flow_to_aq_and_stream})
is that the high degree of connectivity between the streams and the
aquifer in the Scott River system produces balancing counter-forces in
response to hydrologic stresses on the system, such as large recharge
events. This balancing tendency can be temporarily overwhelmed by large
precipitation pulses, but high-flow conditions quickly reduce the volume
of water in the surface water system, returning the Scott River to a
baseline of nearly-balanced stream-to-aquifer and aquifer-to-stream
fluxes. This dynamic also reflects the small size of the available
aquifer storage, relative to the amount of precipitation received by the
watershed in a given water year \citep{DWR2004}.

The resulting water storage limitations mean that multi-year planning,
such as the long-term GSP projects, may be impossible in the Scott River
watershed without making assumptions about how much it will rain
\citeyearpar[i.e., the future climate predictions in Siskiyou
County][]{SiskiyouCounty2021}. If those assumptions are not fulfilled by
future climate, year-by-year adaptive management may be necessary to
achieve management outcomes.

\subsection{\texorpdfstring{\(V_{min.~30~days}\) and \(P_{spill}\)
prediction
utility}{V\_\{min.\textasciitilde30\textasciitilde days\} and P\_\{spill\} prediction utility}}

Though various methods exist to qualitatively predict, in the spring,
the severity of the coming low-flow season in the Scott River watershed,
a quantitative short-term forecasting index could support more rigorous
thresholds for adaptive management. To this end we developed two linear
equations for predicting minimum dry season baseflows about five months
in advance, effectively taking an inventory in each April of relevant
hydrologic inputs. It could be used to support decisions made in the
late spring timeframe regarding the growing season, such as potential
regulatory actions and some farmer cropping decisions.

There are several methods in current use. Observations at existing
monitoring locations, such as weather stations and long-term snow course
records, have been used as ad-hoc hydrologic indices. Historical
adaptive management decisions in the Scott River watershed, such as
planning to purchase surface water rights leases, have relied on
individual monitoring observations, such as percent of snowpack relative
to average conditions, or the Fort Jones flow in the spring
\citep[e.g.,][]{SRWT2018}. Additionally, DWR has effectively extended
the methodology of the SVI and SJI metrics to all of California by
publishing a categorical water year type (WYT) index for all its major
watersheds {[}to the HUC8 level; \citet{DWR2021a}{]}. This metric
quantifies meteoric drought and relies only on precipitation data, so as
to be comparable across the state. Matching SVI and SJI methodology, it
can be calculated at multiple points in each spring, with a final
determination in May, but in the case of Scott Valley it has been used
to classify WYTs only retroactively through 2018. As previously
mentioned it is a relatively complex metric with provisions including a
partial one-year holdover (i.e., dry conditions in the previous year
will make a dry-type categorization more likely the following year), and
non-stationary index thresholds, with a 30-year ranking window.

The proposed quantitative prediction methods map well onto the existing
DWR WYT index, but preserve more detailed information. The primary
advantages of the proposed method over these and other previous methods
of gauging near-term hydrologic conditions is that it is tailored to
local hydroclimate data and is interpretable as a numeric prediction of
fall conditions. This could be used to inform regulatory actions in an
attempt to increase fall environmental flows, or for surface water
diverters to plan for low-flow conditions.

Though it also serves as an indication of the severity of a water year,
the additional specific utility of the second predicted metric,
\(P_{spill}\), may be less than for that of minimum dry season
baseflows. Management decisions such as the last possible date to keep a
temporary stream gauge installed in a river, without risk of it being
washed out, could be informed by a \(P_{spill}\) prediction when
combined with weather forecasts in the fall.

\subsection{Management implications of best-performing predictors}

As described in Results, the linear models that best predicted observed
values of \(V_{min.~30~days}\) and \(P_{spill}\) were both based on the
same two observation locations (the SWJ snow course and the FJRS weather
station; \autoref{fig:watershed_fig_ch3}), both with lengthy observation
records. One interpretation of these results is that the climate inputs
produce a predictable fall watershed response, and that human management
decisions have a negligible influence on fall river flow. However, model
simulations suggest that the timing and magnitude of fall flow increases
can be influenced by human water use {[}e.g., scenarios in Chapter 2 of
this dissertation; Siskiyou County -\citet{SiskiyouCounty2021}{]}.

Multiple possible explanations could reconcile these two pieces of
seemingly contradictory evidence. First, random variability in human
water use could be a contributing factor to the error of the predictive
models of fall-season hydrologic behavior. Alternatively, human water
uses could be so consistent in response to wet or dry season conditions
that these water uses could be implicitly incorporated into the
predictive models. If adaptive management actions (potentially including
events as diverse as regulatory curtailments or individual cropping
choices) are carefully recorded in the future, they could be compared to
residuals of the climate-based predictive models to evaluate whether any
signal of a response to human interventions can be observed.

\subsection{Influences of climate change on predictive indices}

Both predictions (using the full record of hydrologic data) assume some
degree of hydroclimate stationarity, in that it uses historical
snowpack- and precipitation-runoff relationships to predict modern
runoff. In one sense, a longer-term record can be an asset, in that it
provides context for the severity of the dry periods of the past two
decades. In another sense it is a liability for prediction accuracy: for
example, the predicted \(V_{min.~30~days}\) values based on the full
record appear to systematically overpredict \(V_{min.~30~days}\) in the
most recent era (2001-2021; \autoref{fig:two_predictor_model}, top left
panel, and \autoref{fig:v_min_over_time}, middle panel). This suggests
that factors not captured in these climate data, such as warmer air
temperatures, changing upland vegetation and evapotranspiration
dynamics, and possibly unknown changes in water use, may be altering the
relationship between the spring water supply and dry season baseflow
rates.

\section{Conclusions}

This study proposed two locally-tailored hydrologic decision-support
metrics for the Scott River watershed in northern California. Both use
snowpack and cumulative precipitation data from October-April to predict
the quantity of interest: the first is the minimum 30-day flow volume in
a given water year, referred to as \(V_{min,~30~days}\), which typically
occurs in September or October. The second index is the cumulative
rainfall needed to ``fill'' the watershed after the end of the dry
season to a ``spilling'' condition that responds quickly to
precipitation events, referred to as \(P_{spill}\). Both indices can be
calculated at the end of April to support near-term (seasonal) adaptive
management regarding the growing season or the fall, similar to the SVI
and SJI in California's Central Valley. However, climate change may
reduce the predictive accuracy of indices based on long-term data
records, and updates based on smaller numbers of more recent water years
should be considered periodically.

The management choices facing local managers in this basin are difficult
to quantify and summarize, as is the case in basins throughout
California and arid regions globally. Locally-derived summary metrics,
tailored to regional hydrologic dynamics, have provided and will
continue to provide tools for supporting those choices and communicating
them to diverse stakeholders and the general public.



\codedataavailability{Analyses and figures in this manuscript were
drafted in RMarkdown. The RMarkdown scripts are available on the
corresponding author's GitHub page. All data used in this manuscript are
publicly available on local, state or federal data
portals.} %% use this section when having data sets and software code available



%%%%%%%%%%%%%%%%%%%%%%%%%%%%%%%%%%%%%%%%%%
%% optional

%%%%%%%%%%%%%%%%%%%%%%%%%%%%%%%%%%%%%%%%%%

%%%%%%%%%%%%%%%%%%%%%%%%%%%%%%%%%%%%%%%%%%

%%%%%%%%%%%%%%%%%%%%%%%%%%%%%%%%%%%%%%%%%%
\competinginterests{The authors declare no competing
interests.} %% this section is mandatory even if you declare that no competing interests are present

%%%%%%%%%%%%%%%%%%%%%%%%%%%%%%%%%%%%%%%%%%

%%%%%%%%%%%%%%%%%%%%%%%%%%%%%%%%%%%%%%%%%%
\begin{acknowledgements}
This manuscript emerged from dissertation work funded by Siskiyou County
SGMA planning grants, with funding from California water bonds.
\end{acknowledgements}

%% REFERENCES
%% DN: pre-configured to BibTeX for rticles

%% The reference list is compiled as follows:
%%
%% \begin{thebibliography}{}
%%
%% \bibitem[AUTHOR(YEAR)]{LABEL1}
%% REFERENCE 1
%%
%% \bibitem[AUTHOR(YEAR)]{LABEL2}
%% REFERENCE 2
%%
%% \end{thebibliography}

%% Since the Copernicus LaTeX package includes the BibTeX style file copernicus.bst,
%% authors experienced with BibTeX only have to include the following two lines:
%%
\bibliographystyle{copernicus}
\bibliography{library.bib}
%%
%% URLs and DOIs can be entered in your BibTeX file as:
%%
%% URL = {http://www.xyz.org/~jones/idx_g.htm}
%% DOI = {10.5194/xyz}


%% LITERATURE CITATIONS
%%
%% command                        & example result
%% \citet{jones90}|               & Jones et al. (1990)
%% \citep{jones90}|               & (Jones et al., 1990)
%% \citep{jones90,jones93}|       & (Jones et al., 1990, 1993)
%% \citep[p.~32]{jones90}|        & (Jones et al., 1990, p.~32)
%% \citep[e.g.,][]{jones90}|      & (e.g., Jones et al., 1990)
%% \citep[e.g.,][p.~32]{jones90}| & (e.g., Jones et al., 1990, p.~32)
%% \citeauthor{jones90}|          & Jones et al.
%% \citeyear{jones90}|            & 1990


\end{document}
